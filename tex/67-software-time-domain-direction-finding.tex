\section{Time Domain Direction Finding}

\subsection{Testing}
Measure propagation delay of cables on VNA. Screenshot.
Put same cables onto system and look at delay according to system. 
Profit.

\subsection{Time Domain Correlation}
Numpy correlate takes in two vectors, \(\vec{a}\) and \(\vec{b}\) and outputs a vector equal to the correlation of the two, \(c_{ab}\) which is implemented as:
\begin{equation}
  c_{ab}[k] = \sum_{n = 0}^{N-1} a[n+k] * v[n]
\end{equation}

Hence, \(a\) has to be larger. It gets zero padded. 
If the peak occurs after the midpoint, \(a\) had to be articifically advanced to get the signals to correlate meanining it's actually delayed. Occurs AFTER \(v\). 
Hence, a positive (past midpoint) correlation peak implies that \(a\) is delayed compared to \(v\). Negative means that \(a\) is ahead of \(v\), ie: \(a\) occurred BEFORE \(v\). Before means it's more to the left on the plot.


\subsection{Upsampling}
Use the word 'interpolate' somewhere...
Can upsample. Why? Band limited signal sampled above Shannon/Nyquist sampling requirement. Hence we have all possibe information. Hence we're allowed to interpolate as much as is wanted. 
Want to upsample to provide better resolution. Note accuracy does not go up, it's dependant on sample rate, jitter, ENOB etc. But resolution can be arbitrarily increased by upsampling.

Upsampling of large signals can consume lots of CPU cycles and slow down the DF system. However we can exploit the properly that upsampling of the raw time domain signals has the same effect as upsampling the output of the time domain cross correlation. Impulses of even a few microseconds will be tens of thousands of ADC samples which can be expensive to upsample. However the time domain cross correlation output will be much shorter as it's limited to only the correlation interval of interest which corresponds to the number of samples it takes for a signal to propagate from one antenna to the other. Typically only a few samples.

Two ways. More intuative: upsample signal then cross correlate. 
Or cross correlate then upsample. Second is FAR more efficient computationally. 
Time is faily linear with x then upsample. Time is very dependant on signal length with first. Assume it has to do with resample being based on FFT and FFT performing differently with different signal lengths. With x then upsample, signal lenght to be upsampled known before hand.
Upsample then X much more expensive as it needs to upsample the full signal and then X on the huge resulting signal.
X first does only a few steps of X, producing only a few output vectors then upsamples each of those.

Resample first:
Depending on signal length which is controlled by length of pulse, not deterministic, the FFT is \(\Omega(N\log{N})\) to \(\mathcal{O}(N^2)\). Next stage, cross correlation does a whole lot of point MACs. \(\mathcal{O}(N)\). This is dominated by the resample phase. 

X first: Does \(\mathcal{O}(N)\) MACS. Result is a small output vector. Not dependant on N. Hence the resample stage becomes \(\mathcal{O}(1)\). 

Perhaps look at this again for N being the amount of upsampling being done rather than signal length. Or amount of padding?

Table: Variable: (sig len, upsampling, padding). Best case O. Worst case O. 

A few pictures here:
Non upsampled and upsampled correlation.
\begin{figure}
  \centering
  \begin{subfigure}[b]{0.49\textwidth}
    \centering
    \includegraphics[width=\textwidth]{noise-no-upsampled}
    \caption{a}
  \end{subfigure}
  \begin{subfigure}[b]{0.49\textwidth}
    \centering
    \includegraphics[width=0.92\textwidth]{noise-with-upsampled}
    \caption{b}
  \end{subfigure}
  \caption{Bar}
\end{figure}

\begin{figure}
  \centering
  \includegraphics[width=\textwidth]{time-domain-cross-raw-vs-upped}
  \caption{Bar}
  \label{fig:software-aseaweawea}
\end{figure}

\begin{figure}
  \centering
  \begin{subfigure}[b]{0.49\textwidth}
    \centering
    \includegraphics[width=0.95\textwidth]{time-delay-through-full-rf-chain}
    % left, bottom, right, top
    \caption{Phases before calibration}
  \end{subfigure}
  \begin{subfigure}[b]{0.49\textwidth}
    \centering
    \includegraphics[width=0.95\textwidth]{time-delay-in-phase}
    % left, bottom, right, top
    \caption{THIS IS WRONG. GENERATE AND USE CALIBRATED IMAGE HERE.}
  \end{subfigure}
  \caption{Bar}
\end{figure}

\subsection{Calibration}
Different RF chains may have different propagation delays. 
Inject broad band signal into each baseline. Should be zero delay between each. 
If delay is non-zero, write it to calibration file.
For future measurements, subtract this calibration offset from measured delay. 

Here: image of cross correlations of all pre-cal baselines. Single plot. Use long signals for high and narrow peak.
There is another cal section. Merge?

\subsection{De-dispersion}
We're assuming constant propogation delay across frequency. VNA shows mostly correct, but not perfectly. Also other systems have have different dispersion factors. 
De-dispersion. 
More difficult but: single set of coefficients would be used from both time and frequency systems. 
Need a perfect pulse.
Would be very useful to following systems which use the pulses for RFI classification. 
For now: folded dipole antennas should provide minimal dispursion. 
VNA measurements shows cables not very dispersive.
