\chapter{Background to Radio Direction Finding}
As per Small aperture radio df:

The systems mentioned here will be discussed in more detail later in (some section)



Over the next decade, investigation into using loop antennas was done. In 1920 the U.S set up a network of direction finders for marine navigation based on loop antennas.

Poor error tolerance of loop antennas to ionospheric reflections lead to the development of the Adcock array in 1918. 
In 1926 the Watson-Watt DF system was developed which allowed information to be presented to the user on a CRT display. 

By the start of  the World War II, direction finders could operate up to 30 MHz, although they still were subject to problems caused by ionospheric reflections and multipath in general. 
The tactical advantage which DF systems provided prompted significant development into DF during the war. `
