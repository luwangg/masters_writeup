\chapter{Array Design and Construction}
\setsvg{svgpath=./img/array-construction/}
\graphicspath{{./img/array-construction/}}

Although 5 would be best, limitation of 4 ADC inputs.
We've seen that 4 circular have many lines of symetry and hence lots of ambiguities.
Solution: deform the array to remove the ambiguities.
Deforming is a tradeoff.
Circular has equal accuracy from all angles. Deformed will have better performance from certain angles and worse from others.
Also, circular means that elements are all spaced as far apart as possible. Deforming will bring the elements closer together.

\section{Deforming Array}
Want one of the baselines to be similar to a 5-element array baseline. Selected \SI{0.48}{\meter} as that will provide only a single ambiguiy up to just over \SI{300}{\mega\hertz}.
Other baselines to be made long to allow lower coupling and higher accuracy / sensitivity
Photos of how it was deformed.
Plots from the coordinate generator. Explanation of how coordinate generator works.

\section{Circular vs Deformed Ambiguities}
Plot both the fixed angle, varing frequency and the fixed frequency varying angle plots.

\begin{figure}
  \begin{subfigure}{\textwidth}
    \centering
    \includegraphics[width=\textwidth, clip=true, trim = 0 15 53 0]{4-element-circular-ambiguity-over-frequency}
    \caption{Foo}
  \end{subfigure}
  \begin{subfigure}{\textwidth}
    \centering
    \includegraphics[width=\textwidth, clip=true, trim = 0 15 53 0]{4-element-circular-ambiguity-vs-phi}
    \caption{Foo}
  \end{subfigure}
  \caption{Foobar}
\end{figure}
\begin{figure}
  \begin{subfigure}{\textwidth}
    \centering
    \includegraphics[width=\textwidth, clip=true, trim = 0 15 53 0]{4-element-deformed-ambiguity-over-frequency}
    \caption{Foo}
  \end{subfigure}
  \begin{subfigure}{\textwidth}
    \centering
    \includegraphics[width=\textwidth, clip=true, trim = 0 15 53 0]{4-element-deformed-ambiguity-vs-phi}
    \caption{Foo}
  \end{subfigure}
  \caption{Foobar}
\end{figure}

\begin{figure}
  \begin{subfigure}{\textwidth}
    \centering
    \includegraphics[height=0.23\textheight]{df-simulation-circular-0-1}
  \end{subfigure}\\[1em]
  \begin{subfigure}{\textwidth}
    \centering
    \includegraphics[height=0.23\textheight]{df-simulation-circular-0-3}
  \end{subfigure}\\[1em]
  \begin{subfigure}{\textwidth}
    \centering
    \includegraphics[height=0.23\textheight]{df-simulation-circular-0-6}
  \end{subfigure}\\[1em]
  \begin{subfigure}{\textwidth}
    \centering
    \includegraphics[height=0.23\textheight]{df-simulation-circular-1-0}
  \end{subfigure}
  \caption{Foobar}
\end{figure}
\begin{figure}
  \begin{subfigure}{\textwidth}
    \centering
    \includegraphics[height=0.23\textheight]{df-simulation-deformed-0-1}
  \end{subfigure}\\[1em]
  \begin{subfigure}{\textwidth}
    \centering
    \includegraphics[height=0.23\textheight]{df-simulation-deformed-0-3}
  \end{subfigure}\\[1em]
  \begin{subfigure}{\textwidth}
    \centering
    \includegraphics[height=0.23\textheight]{df-simulation-deformed-0-6}
  \end{subfigure}\\[1em]
  \begin{subfigure}{\textwidth}
    \centering
    \includegraphics[height=0.23\textheight]{df-simulation-deformed-1-0}
  \end{subfigure}
  \caption{Foobar}
\end{figure}

\section{Inter-element Coupling}
Those plots from Thomas showing that elements close to each other couple.

