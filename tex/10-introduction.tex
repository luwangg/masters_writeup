\chapter{Introduction}
\label{ch:introduction}
\section{Background}
MeerKAT is a 64-dish radio telescope, aimed at being the precursor to the full \gls{ska} radio telescope. At the time of writing, the KeerKAT array is in the process of being deployed in the Norther Cape of South Africa.
MeerKAT is designed to be a highly sensitive telescope, able to detect very weak signals. The first phase of MeerKAT will receive L-Band signals, from \SI{1.00}{\giga\hertz} to \SI{1.75}{\giga\hertz}. The majority of the dishes are concentrated in the core of the array, an area about \SI{1}{\kilo\meter} in diameter.

Like any radio telescope, MeerKAT requires a quite an RF-quiet environment.
The presence of \gls{rfi} on site would interfere with the ability of the telescope to do science. At the very least, RFI would swamp the weak signals which the telescope is attempting to receive, or  worse, strong RFI could destroy the sensitive amplifiers and digitisers designed to cope only with extremely low power signals.
It is for this reason that the array is being deployed in the Karoo, well away from built up areas and the RF signals that go along with built up areas. 

However, there is always the possibility that RFI will manifest. 
Although all equipment taken to site is tested for RFI before hand, if equipment malfunctions or is not configured correctly or shielded correctly, RFI may be introduced. Additionally, electrical or electronic systems away from site in nearby towns or farms may inadvertently emit RFI. RFI could come from a wide range of sources. Typical examples are communications systems or electronics which emit a well defined narrow band signal, or alternatively from malfunctioning electronics which could emit spark-like discharges producing very short burst of broadband noise.
As such, RFI management systems must be in place to aid in the detection and amelioration of RFI in order to allow the telescope to function optimally.

The purpose of this research is to design a device capable of detecting RFI and performing \gls{aoa} parameter estimation on the detected signals.
The process of estimating the AoA of a signal is known as \gls{df}.
Knowing the \gls{aoa} will allow the SKA team to track down there the rogue RFI signals is coming from and silence it.

\begin{figure}[hb]
  \centering
  \includegraphics[width=0.70\textwidth]{introduction/meerkat-array}
  \includegraphics[width=0.24\textwidth]{introduction/meerkat_dish}
  \caption{Left: photo of the core showing some of the currently deployed MeerKAT dishes in the Karoo. Right: a close up of one of the MeerKAT antennas. Src: \cite{skasawebsite}}
\end{figure}

\section{User Requirements}

The following user requirements were drawn up by three senior engineers at SKA SA: the Director of Science and Engineering, Prof Justin Jonas, Systems Engineer for Infrastructure, Carel van der Merwe and digital backend DSP specialist, Dr Jason Manley. 

\begin{enumerate}
  \item A system to perform direction finding of both impulsive and continuous wave (CW) RFI sources is to be designed.
  \item They key deliverables of this project are a software package and a thesis report.
  \item The software package should have the following functions:
    \begin{enumerate}
      \item It should take in input from a correlator. This could either be time domain cross correlation for impulsive sources or frequency domain cross correlation for continuous sources. 
      \item It should parse a configuration file which contains information about the array configuration and information about the output from the correlator.
      \item The data from the correlator should be used to ascertain the direction of the detected signals.
      \item The software should be designed to fit into a system which has a 100\% \gls{poi}
    \end{enumerate}
  \item The software should have the following user interface features:
    \begin{enumerate}
      \item The user should connect to it via a web interface.
      \item A streaming waterfall plot of frequency vs amplitude should be displayed to the user to serve as monitoring of the RFI environment.
      \item The user should be able to select a band of interest from the waterfall plot.
      \item The direction finding should then be computed for the signal in that band.
      \item The result of the DF should be presented to the user. An investigation must be done into the best way to present this information to the user.
      \item Where appropriate, additional meta information should be displayed to the user, such as measurement accuracy or signal strength.
    \end{enumerate}
  \item The system should be designed to find terrestrial RFI sources.
  \item The system should be designed to be location independent. It could either be deployed to a fixed location or as a mobile device deployed on a vehicle.
  \item This project should be able to interface easily with other systems requiring its data. Specifically, it should be designed to interface with and pass its data on to an allied project which is doing classification of RFI.
  \item The system should be real-time, where real-time is defined as having a latency in the order of a few seconds from receiving signals to displaying results to the user. 
  \item The hardware and software used should be in line with what is used at MeerKAT. This implies the ROACH platform for hardware, Python for back end software and JavaScript for front end software.  
  \item This system must operate in the context of the MeerKAT site, implying the following:
  \begin{enumerate}
      \item In general, the RF environment is sparse. While there will be multiple simultaneous transmitters, it can be assumed there will only be one transmitter in a channel and one source of transients at a given time.
      \item The sources of the emissions will be relatively slow moving, up to the maximum speed of a vehicle on a dirt road; \SI[per-mode=symbol]{60}{\kilo\metre\per\hour}.
  \end{enumerate}

  \item Once the software has been completed, its performance on real life data should be quantified in the following way:
    \begin{enumerate}
      \item A prototype-stage 4-element antenna array should be connected to a \SI{400}{\mega\hertz} baseband digitiser and correlator.
      \item The correlator need not be real time for the demonstration.
      \item As the goal of this project is not to develop a hardware system, there is no specific requirement on receiver sensitivity or noise figure. Whatever the best available hardware is should be used for the antennas, front end and digitiser. 
      \item The performance of the hardware used should be analysed. 
    \end{enumerate}
  \item Mitigation of the effects of performance degradation due to multipath is outside of the scope of this work.

  \item The report produced should contain a theoretical analysis of the performance of the system, as well as an analysis of the performance of the prototype on site with real signals. 
\end{enumerate}

\section{Requirements Review}
The requirements as stated by the \gls{ska} are clear and understood. Some notable implications from the requirements:

\begin{itemize}
  \item The system which will be designed here will probably not be immediately applicable to combating RFI at the MeerKAT site. The reason is that this project is focused on the \gls{df} implementation, not hardware. Hence no down converter will be designed in meaning the frequency band being digitised by this system will not overlap the MeerKAT band. 
    It will be up to future work to add a suitable \gls{rf} front end in order to allow the system to operate in the MeerKAT band of interest. However, the implication for this work is that it is essential that this system be designed such that it is trivial to reconfigure the system for a different RF front end and frequency band. 

  \item Based on the fact that the emitters will be either stationary or slow moving, it should be acceptable to have integration times in the order of 1 second. 

  \item The requirement of a 100\% \gls{poi} will impose restrictions on what antenna array and receiver can be used, and this will in turn have restrictions on the specific \gls{df} algorithm which can be used. 

  \item Seeing as this system is intended to be used by \gls{ska} and will require future work to make it production ready, the source code should be structured such to allow collaborative development and located somewhere accessible. Github will be used. Following is a list of code repos used in this project. 

\end{itemize}

\begin{table}
  \centering
  \begin{tabu}{c|c}
    Description & URL \\
    \hline
    This document & \url{https://github.com/jgowans/masters_writeup} \\
    Ambiguity simulation & \url{https://github.com/jgowans/phase_ambiguity} \\
    iADC calibration & \url{https://github.com/jgowans/iADC_calibration} \\
    Direction finder front end & \url{https://github.com/jgowans/directionFinder_web} \\
    ROACH firmware & \url{https://github.com/jgowans/correlation_plotter} \\
    Direction finder backend & \url{https://github.com/jgowans/directionFinder_backend} \\
  \end{tabu}
  \caption{Github repos for this project}
  \label{tab:lit-review-repos}
\end{table}

\section{Report Outline}
This, Chapter 1, has contained the problem statement and user requirements of the system which needs to be designed an implemented. 

Next in \Cref{ch:lit-review}, the existing literature around DF systems will be explored. This will serve as the foundation for the system being designed here. The exploration will focus on direction finding fundamentals and a comparison of algorithms. It will examine DF for both continuous narrow band signals and impulsive signals. It will explore detection algorithms for impulsive sources, as well as calibration techniques and error calculations. The outcome of this chapter should be sufficient information to make informed system design decision.

\Cref{ch:system-design} contains the high level system design, showing what blocks need to be designed and how they will be linked together in order to build a complete system. Here, the system is broken into three main components: the RF front end, the ROACH digitiser and DSP engine, and computer software for final AoA computation and logging. Additionally, the exactly DF algorithm being used by the system will be defined and simulations of its performance run.

\Cref{ch:rf-front-end} discussed the antenna array and RF front end. It details designing and building the 4-element antenna array, as well as assembling and measuring a board of low noise amplifiers and filters.

\Cref{ch:firmware-design} discusses the implementation of the FPGA firmware to do data acquisition and the first stage of DSP. The focus is on designing the FPGA sub-system in simulink and demonstrating it being functional both in simulink simulations and, most importantly, running on hardware and processing real signals in the lab.

\Cref{ch:software-design} details Python code which implements the direction finding algorithms. It requires interfacing the computer with the ROACH to read out data, creating a model of the antenna array, and estimating the AoA based on the received data and the array model. It also has to handle calibration, logging, and showing signal plots to give the operator insight into the signals being received.

\Cref{ch:field-trials} reports on the field trials of the DF system. There are two field trials sessions: one early in the project with a basic system, to get a feel for the types of RFI data. Another with the final system doing real DF measurements and checking the accuracy of the system.

\Cref{ch:conclusions} contains conclusions indicating what has been achieved in the project and comments on the system which has been implemented. Also, scope for future work is looked at.

