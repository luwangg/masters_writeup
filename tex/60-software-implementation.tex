\chapter{Software}
\label{ch:software-design}
\setsvg{svgpath=./img/software/}
\graphicspath{{./img/software/}}
Software was written in Python to run on a computer which is connected to the ROACH via Ethernet. The software has two broad functions: to interact with and provide an abstraction to the FPGA, and to perform angle of arrival estimation based both on the received signals and on the antenna model. This Chapter looks at the structure of the software package and some results indicating it is working as expected.

\section{Code Structure}
\begin{figure}
  \centering
  \makebox[\textwidth][c] { 
    \includegraphics[width=1.1\textwidth, clip=true, trim = 80 365 80 80]{backend-arch}
    % left, bottom, right, top
  }
  \caption{UML class diagram of software structure in \lstinline{DirectionFinder-backend}.}
  \label{fig:software:df-backend-uml}
\end{figure}
The code had to be designed in accordance with good object-orientated methodologies in order to provide a useful, well defined and easily extendible interface to the various software components which need to interface with one another and with the correlator. As such, there was significant emphasis encapsulating logic into classes which mirrored the physical structure of the correlator and the antenna array with regard to modularising key components and writing reusable code.

The main package containing the array modeling, the ROACH interface and the AoA estimation is \lstinline{DirectionFinder-backend}\footnote{\url{https://github.com/jgowans/directionFinder_backend}}. The structure of the classes in this package is shown as a UML diagram in \Cref{fig:software:df-backend-uml}. The root class, \lstinline{DirectionFinder} is composed of an \lstinline{AntennaArray} and a \lstinline{Correlator}. 

The \lstinline{AntennaArray} is initialised from an array geometry file that has been produced by the element coordinate calculator discussed in Chapter 3. It is able to return expected baseline phase shifts or time delays for all antenna pairs at any angle. The returned values are used to get the theoretical array response for a given angle at a specific frequency. Due to the modular structure of the \lstinline{AntennArray} class, it can be used to simulate and provide information about an antenna array with any number of elements in any configuration. This is an example of the general purpose nature of the application which is being developed here.

The \lstinline{Correlator} provides an interface to processed data from the FPGA. It is able to fetch both frequency domain cross correlations from the snapshot blocks as well as time domain snapshots which it gets by reading raw time domain data and doing in-memory time domain cross correlations. The \lstinline{Correlator} also contain a \lstinline{ControlRegister} which abstracts the raw bit or word read/writes necessary to interact with the status and configuration registers. Once again the software has been designed with generality in mind: any number of baselines can be read out, having spectrums with any number of points for arbitrary start/stop frequencies. These are initialisation configuration parameters of the classes providing the abstractions.\\

The software application is launched by a Python executable which takes command line arguments that define the IP of the ROACH, the path to the configuration file for the ROACH, the frequency domain start/stop frequencies, the frequency or frequency range which should be monitored for direction-finding, impulse setpoint, accumulation length, and a comment string to label the observation which is being made. Once running, the software will monitor for impulses and it will continuously DF the narrow band signal. The raw baseline measurements as well as the calculated AoA are both printed to the screen and are written to a log file in a format which can easily be post-processed for plotting. Warnings such as overflow are printed to the screen.\\

The following sections explain how the software goes about ascertaining angle of arrival for time and frequency domain signals.

\section{Frequency Domain Direction Finding}
The frequency domain DF algorithm assumes that the signal of interest may well be below the noise floor. As such, no detection is done; the direction finding is continuous. 

When the application starts up, it reads in the array configuration file and constructs an \lstinline{AntennaArray} object from it. It then samples the antenna array manifold vector at \SI{0.36}{\degree} intervals, corresponding to 1000 baseline phase shift vectors. These are stored in a hash in memory. 

Next, the ROACH is initialised by writing to the accumulation length (\lstinline{acc_len}) register and pulsing the \lstinline{sync} line. The accumulation length value is set based on user input, typically it will be around 1 second. Next, it constructs instances of a \lstinline{Snapshot} class, one for each baseline cross correlation BRAM snapshot on the FGPA. These snapshot blocks are all armed, and once armed the snapshot trigger ungated. The system sits in a tight loop watching the accumulation counter. When the accumulation counter ticks, the snapshot trigger is gated and all snapshot values read out. Calibration factors (discussed more later) are then applied to the signals. The strongest signal in the frequency interval of interest is found, and the phase shifts as seen by each baseline correlation are fetched and stored in a map of baseline name (eg: ``1x3'')  to phase shift. Finally, the \lstinline{DirectionFinder} iterates through all 1000 simulated angles, comparing the simulated baseline phase shifts to the observed baseline phase shifts and selects the angle whose simulated values most closely match the observed values. In pseudo code:

\begin{lstlisting}
AntennaArray antenna_array = new AntennaArray(config_file)
Correlator correlator = new Correlator(...)

while True:
  Correlation correlation = correlator.get_correlation()
  Float frequency = correlation.strongest_signal_in_range(f_start, f_stop)
  Map<Baseline, Float> observation = correlation.baseline_phase_shifts_at(frequency)

  Float best_angle = 0;
  Float lowest_difference = INT_MAX;

  for angle in antenna_array.sampled_angles
    Float difference = antenna_array.manifold_at(frequency, angle).compare(observation)
    if difference < lowest_difference:
      best_angle = angle;
      lowest_difference = difference
  write_result(best_angle)
\end{lstlisting}

The \lstinline{compare} method computes the difference between vectors by the RMS of the difference of their terms. When the difference, \(d\) is computed for two vectors \(\vec{a}\) and \(\vec{b}\) with \(k\) terms, it uses arctan to account for the fact phase wraps around \(2\pi\), as follows:

\begin{equation}
  d = \left[ \sum_{i=0}^{k}\abs{\atantwo(\sin(\vec{a}_i - \vec{b}_i), \cos(\vec{a}_i - \vec{b}_i))}^2 \right]^{1/2}
\end{equation}


\section{Time Domain Direction Finding}

As mentioned in Chapter 3, the algorithm behind time domain DF is remarkably similar to frequency domain DF. The main difference is that the comparison method uses baseline time delays rather than phase shifts. 

One important difference is that where the frequency domain cross-correlations are calculated using DSP on the FPGA, the time domain cross-correlations need to be calculated on the computer. This is not a hard requirement, but the implementation of the FPGA design did not cater for time domain cross-correlations which can be a lot trickier to do in hardware since the length of the pulse is dynamic.

The time domain cross-correlation is implemented, as defined, by multiplying each point of one antenna's signal \(\vec{a}\) by the corresponding points of a shifted version of another antenna's signal \(\vec{b}\) for some shift \(k\), producing the correlation \(c_{ab}(k)\):
\begin{equation} \label{eq:software:time-domain-cross-correlation}
  c_{ab}(k) = \sum_{n = 0}^{N-1} a_{n} b_{n+k}
\end{equation}

The range of values for \(k\) will be picked to span an interval a bit larger than the propagation time of a signal over the whole array. For the array being used here this will correspond to \(k\in [-100, 100]\), \(k\in \mathbb{Z}\)\footnote{This is about 10 times more than is necessary, but seeing as the correlation output will be upsampled the signal should not be short, or the frequency resolution available in the upsampling process will be limited, leading to an inaccurate output.}. To make this possible, \(\vec{b}\) is zero-padded with 100 zeros on each end. Hence, in \Cref{eq:software:time-domain-cross-correlation} the value of \(N\) is the number of points in the shorter, non-padded vector \(\vec{a}\).

Once an array of \(c\) values for various \(k\) values has been generated, the \(c\) array is upsampled using the Fourier method discussed in Chapter 3.  Upsampling of large signals can consume lots of CPU cycles and slow down the DF system. However by upsampling after cross-correlation we take advantage of the property that upsampling the raw time domain signals has the same effect as upsampling the output of the time domain cross-correlation. While impulses of even a few microseconds will be tens of thousands of ADC samples (which can be expensive to upsample), the time domain cross-correlation output will be only 200 samples as it's limited to the correlation interval. 

Once the upsampling is complete, calibration factors are applied to \(c\), then the maximum value of the upsampled calibrated \(c\) is found and the corresponding time shift noted. This is done for the snapshot of each combination of pairs of antennas, \(a\) and \(b\). The result is a map of baselines to observed time differences seen by that baseline.

The same algorithm as the frequency domain DF is then used to find the angle where the simulated baseline time delays most closely match the observed ones. That best-matching angle is then chosen as the angle or arrival.




\section{Calibration}


Although efforts have been made to keep the RF chains for each antenna's time/phase delay matched, it hasn't always been possible to do so. Small errors in measured phase/time shifts can throw off the measurement results. Examples of where mismatches can occur are:

\begin{itemize}
  \item Antenna cables: the antennas which were purchased for this project do not have the same length cable length coming out of them. This means that although the antennas may receive a signal with a certain phase difference, a different phase difference is emitted from the cables as the propagation delay is different.
  \item RF front end: the RF components and the connecting cables from the antennas to the ADCs may not be matched. Measurements in earlier chapters have shown that the difference in this setup is low, but it is worth catering for this mismatch anyway for future systems which may have RF front ends with more significant mismatches.
  \item ADC cores: The ADCs should be clocked exactly in phase, but mismatches in clock distribution or internal ADC characteristics cause the actual samples not to be exactly in phase.
\end{itemize}

Following is a brief description of how these mismatches were measured and calibrated out in software.

\subsection{Antenna cable lengths}
It is difficult to empirically measure the differences in signal propagation delay through the antennas as delayed by their cables. To cater for cable length mismatches in a generic way, the direction finding system takes as input a file specifying the measured cable lengths coming out each of the antennas as well as specifying the velocity factors of the cable. The velocity factor is a property of the specific cable type in use. In this case, the RG214 \SI{50}{\ohm} cables used here have velocity factor of 0.6 as per their data sheet. \Cref{tab:software-antenna-cable-lengths} shows the measured length of cables from each antenna. The software then calculates the resultant time delay \(t\) as follows:
\begin{equation}
    t = \frac{l}{\text{vf} \times c}
\end{equation}
where \(l\) is the length of the cable, \(\text{vf}\) is the velocity factor and \(c\) the speed of light. Phase shift at a particular frequency can easily be deduced from time delay.

To verify the implementation of this technique in a laboratory test, all ADC channels were connected to the same noise source via equal-length cables except channel 0 which had a longer cable. The difference in propagation delay between the longer and shorter cables was checked on a network analyser. \Cref{fig:software-two-cables-vna} indicates that cable length response was flat and that the actual difference between the propagation delays is \SI{2.49}{\nano\second}. Next, the DF system was used to do a time domain cross correlation before and after calibration factors were applied. \Cref{tab:software-cable-lenth-compensation} indicates that after cable length calibration factors had been applied, the system's measurements accurately reflected actual time differences and it correctly identified that the signals were arriving in phase.

\begin{table}
  \centering
  \begin{tabu}{c|c}
    Antenna Number & Cable Length (m)\\
    \hline
    0 & 0.557 \\
    1 & 0.566 \\
    2 & 0.510 \\
    3 & 0.590
  \end{tabu}
  \caption{Lengths of cables coming out of antennas}
  \label{tab:software-antenna-cable-lengths}
\end{table}

\begin{figure}
  \centering
  \includegraphics[width=0.6\textwidth]{two-cables-vna}
  \caption{Two different length cables on VNA doing time measurement showing propagation delay difference of \SI{2.49}{\nano\second}}
  \label{fig:software-two-cables-vna}
\end{figure}

\begin{table}
  \centering
  \begin{tabu}{c|r|r}
    Baseline & Uncalibrated (ns)& Calibrated (ns)\\
    \hline
    0x1 & 2.492 & 0.002 \\
    0x2 & 2.489 & -0.002 \\
    0x3 & 2.495 & 0.002 \\
    1x2 & -0.001 & -0.002 \\
    1x3 & 0.001  & 0.000 \\
    2x3 & 0.005 & 0.005
  \end{tabu}
  \caption{Time domain correlation peak position for each baseline before and after cable length calibration factors were applied. Channel 0 had a longer cable length which was the network analyser showed was \SI{2.49}{\nano\second} and the correlator produced the same result. ADC sample period: \SI{1.25}{\nano\second}. Upsampled correlation step size: \SI{1}{\pico\second}}
  \label{tab:software-cable-lenth-compensation}
\end{table}

\subsection{RF front end mismatches}
The next class of time/phase mismatches is from the rest of the RF front end: filters, amplifiers, connecting cables and ADCs. This is different to the antenna cables because the mismatches here can be measured by injecting signals into the system. This is how mismatches through the remainder of the RF chain were measured:
\begin{itemize}
  \item For the time domain, broadband noise was injected into the start of the RF chain, raw data captured from the ADC and cross-correlated to find time differences introduced by each input path. The offset of the correlation peak from 0 is the time delay error for that baseline.
  \item For the frequency domain, iteratively through each frequency channel, a tone centered in the frequency channel was injected from a signal generator exactly in phase to each signal path. The phase differences as seen by the output of the accumulator on the ROACH were recorded. These phase differences were the error for the baseline in that frequency channel.
\end{itemize}

These two experiments produced two calibration files, one for broadband time delay and one for phase shift at each specific channel. These two files are then used by direction finding code to subtract the time or phase offset introduced by the RF chain.

\Cref{fig:software:time-domain-cal-graphs} shows the time domain correlation peaks before and after calibration factors are applied, indicating that after calibration the peaks correctly align when broadband noise is injected in phase.

\Cref{fig:software:time-domain-cal-graphs} shows how phase offsets due to differing and non-linear phase performance across frequency is significantly reduced by applying frequency domain calibration factors.

\begin{figure}
  \centering
  \begin{subfigure}[b]{0.49\textwidth}
    \centering
    \includegraphics[width=0.95\textwidth, clip=true, trim = 20 0 40 0]{time-delay-through-full-rf-chain-pre-cal}
    % left, bottom, right, top
    \caption{Before calibration}
  \end{subfigure}
  \begin{subfigure}[b]{0.49\textwidth}
    \centering
    \includegraphics[width=0.95\textwidth, clip=true, trim = 20 0 40 0]{time-delay-through-full-rf-chain-post-cal}
    \caption{After calibration}
  \end{subfigure}
  \caption{Time domain cross correlation plots of each baseline for broadband noise injected in phase before and after calibration.}
  \label{fig:software:time-domain-cal-graphs}
\end{figure}

\begin{figure}
  \begin{subfigure}[b]{0.49\textwidth}
    \centering
    \includegraphics[width=0.95\textwidth]{freq-shift-full-rf-chain}
    \caption{Before calibration}
  \end{subfigure}
  \begin{subfigure}[b]{0.49\textwidth}
    \centering
    \includegraphics[width=0.95\textwidth]{freq-shift-full-rf-chain-after-cal}
    \caption{After calibration}
  \end{subfigure}
  \caption{Phase shifts through full RF front end of signal being injected exactly in phase before and after calibration.}
  \label{fig:software:time-domain-cal-graphs}
\end{figure}



\section{Summary}
This Chapter has discussed the structure of the direction finding software package, as well as how the angle of arrival estimation calculations were implemented. It started by showing how the class structure mirrors that of the physical system, providing abstractions around antenna array models, FPGA interactions and FPGA data outputs. Emphasis was placed on making the algorithms reconfigurable for any number of inputs or frequency range.

The frequency domain and time domain angle of arrival estimations were both done in a similar way, by correlating the observed baseline time/phase differences with the theoretical time/phase differences for each possible angle of arrival and finding the one which most closely matches. 

Finally, calibration section discussed potential sources of error in the analogue components of the system. It showed how those sources of error were measured and made available to the direction finding code so that they could be subtracted out to provide a more accurate calibrated measurement of the real signals.
