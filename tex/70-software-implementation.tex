\chapter{Software Interface}
\label{ch:software-design}

\section{Code Structure}
It was necessary to write code to allow the computer to interface with the correlator running on the ROACH.
The code had to be carefully designed in accordance with good object orientated design methodologies in order to provide a useful, well defined and easily extendible interface to other code which needs to interface with the correlator.
As such, there was significant emphasis encapsulating logic into classes which mirrored the physical structure of the correlator in the sense of modularising the key components and writing reusable code.
Following is a discussion of the software interface to the correlator with specific focus on the various classes and interfaces.

Instance of a correlator is created.
This then includes:
\begin{enumerate}
  \item An instance of a ControlRegister
  \item Multiple instances of Snapshot classes. One will be created for each snapshot block on the FPGA.
  \item Multiple instance of the Correlation clas, one for each of the cross correlation baselines.
\end{enumerate}

The direction finder code takes an instance of a correlator which can then be used for easy access to the data from the orrelator.

\subsection{Control Register}
The control register naturally lends itself to a class with interfaces to modify the different bit groups of the register in logical ways.
A python module called software\_register was written with a single class: ControlRegister. This class contains a value parameter which mirrors the value programmed into the FPGA's control register. It then has the following interface methods:
\begin{description}
  \item[pulse\_sync] Toggles the syncronisation reset bit low to high and then high to low.
  \item[block\_trigger ; allow\_trigger] These methods either set of clear bit the bit that controls the gate which blocks or allows the snapshot blocks being triggered by a complete accumulation. The idea is that this will be used to allow all of the snapshot blocks to be armed sequentially before they can be triggered all at once.
  \item[reset\_accumulation\_counter] Pulses the bit which resets the counter which increments every time an accumulation completes and triggers the snap blocks. This counter lets us keep track of how many accumulations have been performed by the system.
  \item[pulse\_overflow\_rst] Similar to the above, except this clears the latch which gets latched high whenever the FFT block overflows. This essentially allows us to 'acknowledge' that we have seen the occurrence of the overflow flag.
  \item[select\_adc] Controls which of the four RF inputs is connected to the time domain snapshot block via a multiplexer. By multiplexing the steams, it saves the logic and BRAM of having to have four seperate snap blocks, at the expense of one multiplexer and not being able to synchronously sample the ADC streams in the time domain. This is considered an acceptable tradeoff due to the fact that the time domain snap is only designed to be used for getting an approximation of the signal strength arriving at an antenna and this does not need to be done in parallel for each antenna.
  \item[set\_shift\_schedule] This methods takes a \SI{12}{\bit} number and sets the FFT shift schedule to that number. While this design only has a 10 stage FFT, 12 bits is kept in the register for future expansions.
\end{description}

All of the methods in this class log their actions as well as the new value of the control register when it is modified at the debug log level.

\section{Calibration}


The software needs to have provision to calibrate various aspects:
\begin{itemize}
  \item RF front end: the LPF and LNA and cable from LNA to ROACH may have different delay or phase shift factors. Measurements in earlier chapters show not really different, but worth doing any way for future systems which may have RF front ends with more significant shifts. This can be calibrated very accurately using a known signal though the RF chain and measuring the response.
  \item Same as above but in time domain
  \item Antenna cables: the antennas which were purchased for this project do not have the same length cables. Table (following) shows various lengths. It would be difficult to imperically measure the phase shift caused by these cables. Rather, by knowing the velocity factor through the cable and measuring the length of the cable the difference can be deduced.
\end{itemize}

\subsection{Antenna Cable Length Compensation}
\begin{table}
  \centering
  \begin{tabu}{c|c}
    Antenna Number & Cable Length (m)\\
    \hline
    0 & 0.557 \\
    1 & 0.566 \\
    2 & 0.510 \\
    3 & 0.590
  \end{tabu}
  \caption{Lengths of cables coming out of antennas}
  \label{tab:software-antenna-cable-lengths}
\end{table}

According to the manufacturer, Magnavolt, the cables are all RG214 which has a velocity factor of 0.659. 

Wrote JSON file with this information.

For each baseline, software calculates difference in propagation times for each element of the baseline created by length and velocity factor. Calculates corresponding phase shift for each frequency bin based on this and then aplies this frequency shift to the cross correlation in there.

Here: extract of JSON file.
A bit of maths of what the software does?

Discuss how this applies to both time and frequency domain.

Two cables: \SI{0.512}{\meter} and \SI{1.005}{\meter}. Both RG58. According to datasheet, velocity factor of 0.66.
Theoretical delay:
\begin{equation}
  \begin{split}
    t &= \frac{l}{\text{vf} \times c}\\[1em]
  \therefore \Delta t &= \frac{l_a}{\text{vf}_a \times c} - \frac{l_b}{\text{vf}_b \times c} \\[1em]
                      &= \frac{1.005}{0.66c} - \frac{0.512}{0.66c} \\[1em]
    &= 2.4899 \text{ ns}
  \end{split}
\end{equation}
VNA shows Propagation delay difference of \(\frac{5.2532+5.1916}{2} - \frac{2.7443+2.7206}{2} = 2.4901\) ns. This is a difference of below \SI{0.1}{\percent}.

\begin{table}
  \centering
  \begin{tabu}{c|r|r}
    Visibility & Uncompensated (ns)& Compensated (ns)\\
    \hline
    0x1 & 2.492 & 0.002 \\
    0x2 & 2.489 & -0.002 \\
    0x3 & 2.495 & 0.002 \\
    1x2 & -0.001 & -0.002 \\
    1x3 & 0.001  & 0.000 \\
    2x3 & 0.005 & 0.005
  \end{tabu}
  \caption{ADC sample period: \SI{1.25}{\nano\second}. Upsampled correlation step size: \SI{1}{\pico\second}}
  \label{tab:software-cable-lenth-compensation}
\end{table}

It's clear that the system is able to correctly measure the cable length difference to within \SI{3}{\pico\second} accuracy which is below 1 hundredth of the ADC sample period.
Also it's clear that the system can successfully apply the calibration factors from the JSON file specifying cable length and velocity factor and compensate for the cable length mismatch down to the same level of accuracy: below 1 hundredth of an ADC sample.

\begin{figure}
  \centering
  \includegraphics[width=0.9\textwidth]{two-cables-vna}
  \caption{Two different length cables on VNA doing time measurement showing propagation delay difference of \SI{2.49}{\nano\second}}
  \label{fig:software-two-cables-vna}
\end{figure}


\subsection{RF Chain Compensation (Calibration?) in Frequency Domain}
The LPF + LNA + Long RG58 cable from the front end to the roach introduces difference phase and time shifts. 

\begin{figure}
  \centering
  \begin{subfigure}[b]{0.49\textwidth}
    \centering
    \includegraphics[width=0.95\textwidth]{freq-shift-full-rf-chain}
    % left, bottom, right, top
    \caption{Phases before calibration}
  \end{subfigure}
  \begin{subfigure}[b]{0.49\textwidth}
    \centering
    \includegraphics[width=0.95\textwidth]{freq-shift-full-rf-chain-after-cal}
    % left, bottom, right, top
    \caption{After cal}
  \end{subfigure}
  \caption{Bar}
\end{figure}

\subsection{RF Chain Compensation in Time Domain}

\begin{figure}
  \centering
  \begin{subfigure}[b]{0.49\textwidth}
    \centering
    \includegraphics[width=0.95\textwidth]{time-delay-through-full-rf-chain}
    % left, bottom, right, top
    \caption{Phases before calibration}
  \end{subfigure}
  \begin{subfigure}[b]{0.49\textwidth}
    \centering
    \includegraphics[width=0.95\textwidth]{time-delay-in-phase}
    % left, bottom, right, top
    \caption{THIS IS WRONG. GENERATE AND USE CALIBRATED IMAGE HERE.}
  \end{subfigure}
  \caption{Bar}
\end{figure}

\input{./tex/75-software-array-coordinate-calculator}
\section{Frequency Domain Direction Finding}
For weak, narrow band, continuous signals.
Foo.

\section{Time Domain Direction Finding}

\subsection{Testing}
Measure propagation delay of cables on VNA. Screenshot.
Put same cables onto system and look at delay according to system. 
Profit.

\subsection{Upsampling}
Use the word 'interpolate' somewhere...
Can upsample. Why? Band limited signal sampled above Shannon/Nyquist sampling requirement. Hence we have all possibe information. Hence we're allowed to interpolate as much as is wanted. 
Want to upsample to provide better resolution. Note accuracy does not go up, it's dependant on sample rate, jitter, ENOB etc. But resolution can be arbitrarily increased by upsampling.

Two ways. More intuative: upsample signal then cross correlate. 
Or cross correlate then upsample. Second is FAR more efficient computationally. 
Time is faily linear with x then upsample. Time is very dependant on signal length with first. Assume it has to do with resample being based on FFT and FFT performing differently with different signal lengths. With x then upsample, signal lenght to be upsampled known before hand.
Upsample then X much more expensive as it needs to upsample the full signal and then X on the huge resulting signal.
X first does only a few steps of X, producing only a few output vectors then upsamples each of those.

Resample first:
Depending on signal length which is controlled by length of pulse, not deterministic, the FFT is \(\Omega(N\log{N})\) to \(\mathcal{O}(N^2)\). Next stage, cross correlation does a whole lot of point MACs. \(\mathcal{O}(N)\). This is dominated by the resample phase. 

X first: Does \(\mathcal{O}(N)\) MACS. Result is a small output vector. Not dependant on N. Hence the resample stage becomes \(\mathcal{O}(1)\). 

Perhaps look at this again for N being the amount of upsampling being done rather than signal length. Or amount of padding?

Table: Variable: (sig len, upsampling, padding). Best case O. Worst case O. 

A few pictures here:
Non upsampled and upsampled correlation.
\begin{figure}
  \centering
  \begin{subfigure}[b]{0.49\textwidth}
    \centering
    \includegraphics[width=\textwidth]{software/noise-no-upsampled}
    \caption{a}
  \end{subfigure}
  \begin{subfigure}[b]{0.49\textwidth}
    \centering
    \includegraphics[width=0.92\textwidth]{software/noise-with-upsampled}
    \caption{b}
  \end{subfigure}
  \caption{Bar}
\end{figure}

\begin{figure}
  \centering
  \includegraphics[width=\textwidth]{software/time-domain-cross-raw-vs-upped}
  \caption{Bar}
  \label{fig:software-aseaweawea}
\end{figure}

\begin{figure}
  \centering
  \begin{subfigure}[b]{0.49\textwidth}
    \centering
    \includegraphics[width=0.95\textwidth]{software/time-delay-through-full-rf-chain}
    % left, bottom, right, top
    \caption{Phases before calibration}
  \end{subfigure}
  \begin{subfigure}[b]{0.49\textwidth}
    \centering
    \includegraphics[width=0.95\textwidth]{software/time-delay-in-phase}
    % left, bottom, right, top
    \caption{THIS IS WRONG. GENERATE AND USE CALIBRATED IMAGE HERE.}
  \end{subfigure}
  \caption{Bar}
\end{figure}

\subsection{Calibration}
Different RF chains may have different propagation delays. 
Inject broad band signal into each baseline. Should be zero delay between each. 
If delay is non-zero, write it to calibration file.
For future measurements, subtract this calibration offset from measured delay. 

Here: image of cross correlations of all pre-cal baselines. Single plot. Use long signals for high and narrow peak.
There is another cal section. Merge?

\subsection{De-dispersion}
We're assuming constant propogation delay across frequency. VNA shows mostly correct, but not perfectly. Also other systems have have different dispersion factors. 
De-dispersion. 
More difficult but: single set of coefficients would be used from both time and frequency systems. 
Need a perfect pulse.
Would be very useful to following systems which use the pulses for RFI classification. 
For now: folded dipole antennas should provide minimal dispursion. 
VNA measurements shows cables not very dispersive.

\input{./tex/78-software-logging.tex}
