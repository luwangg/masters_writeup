\chapter{Field Trials}
\label{ch:field-trials}
\setsvg{svgpath=./img/field-trials/}
\graphicspath{{./img/field-trials/}}

In order to test the operation of the complete system, it was necessary to conduct field trials where the DF system was taken from the lab, made into a portable self-contained unit and tested on real signals in the field. There were two field trials done: the first was done early in the project when the system could only do time domain capture from two channels. The second was at the end of the project with the full four channel time and frequency domain system. The objective of the first field trial was to test the data capturing ability of the ROACH as well as to get samples of what real impulsive RFI looks like. The objective of the second field trial was to test the performance and accuracy of the system.

This Chapter first looks briefly at the results of the first field trial and the implications of the types of impulsive RFI that were capture. It then looks at the full system that was tested at the end of the project, tracking both weak narrow band as well as strong impulsive signals. Additionally some calculations and simulations are done around multipath and wave propagation to decide on suitable dimensions for the field trials.


\section{First trial: Sample Impulsive RFI}

A few months after the start of the project, the system consisted of a simple two-element antenna array and a simple ROACH design. The ROACH streamed input from a single two-channel ADC, did threshold detection and stored a small snapshot of raw time domain ADC samples to FPGA Block RAM (BRAM). Computer code was at this point only able to pull raw samples, save them, and do in memory time domain cross-correlation. 

This system was taken to the SKA's MeerKAT site in the Karoo with the objective of getting some snapshots of RFI to see what characteristics the impulses had as well as to attempt impulsive RFI hunting. Two types of antenna were used: printed LPDA antennas and omni-directional conical antennas.
The two antenna types and the setup and operation are shown in \Cref{fig:field-trials:first-system-setup}. Simple power detection was done by FPGA firmware which was designed to detect when the received signal amplitude went above a threshold,\footnote{Threshold was set by sampling the ambient signal amplitude for a few minutes and defining the threshold two standard deviations above ambient.} and then to capturing the pulse to DRAM. Once captured, detection was paused and the computer notified of the pulse. The computer would then read out the samples and re-enable detection when finished. Examples of captured signals of unknown origin are shown in \Cref{fig:field-trials:unknown-time-domain} and signals of known origin in \Cref{fig:field-trials:known-time-domain}. From this, we can see that there are almost no characteristics that all of the impulsive signals had in common. They varied in terms of all key pulse classification parameters: pulse length, shape, frequency content and repetition pattern. This fact contributed to the decision to do power detection for the impulsive signals. Since it would not be practical to attempt any form of matched filtering when hunting pulses that are of unknown origin, all that can be done is to look for a burst of energy in the time domain.

Direction finding from the raw time domain signals was attempted on the computer by doing the time domain cross-correlation process described earlier. However, these attempts to hunt for RFI using the two-element system did not prove successful due to the \SI{180}{\degree} ambiguity inherent in a two-element design and due to calibration techniques not having yet been developed at that early stage of the project. Also, at that stage the system needed to be plugged in to run and hence was constrained to be close to a power source.

\afterpage{
  \thispagestyle{empty}
  \begin{figure}
    \vspace{-4em}
    \centering
    \begin{subfigure}{0.8\textwidth}
      \centering
      \includegraphics[width=\textwidth]{first-trip-yagi-antennas}
      \caption{Two element array of compact printed LPDA antennas.}
    \end{subfigure}
    \begin{subfigure}{0.9\textwidth}
      \centering
      \includegraphics[width=\textwidth]{first-trip-omni-antennas}
      \caption{Two element array of EM-6916 Omni-directional conical antennas.}
    \end{subfigure}
    \begin{subfigure}{0.9\textwidth}
      \centering
      \includegraphics[width=\textwidth]{first-trip-processing}
      \caption{Laptop linked via Ethernet to the ROACH, processing collected data in real time.}
    \end{subfigure}
    \caption{Setup for first field tests at the SKA's MeerKAT site in the Karoo. Early 2014.}
    \label{fig:field-trials:first-system-setup}
  \end{figure}
  \clearpage
}
%TODO: put the following back
%\skiptoevenpage
\begin{landscape}
  \thispagestyle{empty}
  \begin{figure}
  \centering
  \makebox[\textwidth][c]{
    \begin{subfigure}{0.55\textwidth}
      \includegraphics[width=\textwidth]{sample-rfi-site/14-03-13-11-23-59}
    \end{subfigure}
    \begin{subfigure}{0.55\textwidth}
      \includegraphics[width=\textwidth]{sample-rfi-site/14-03-13-11-56-07}
    \end{subfigure}
    \begin{subfigure}{0.55\textwidth}
      \includegraphics[width=\textwidth]{sample-rfi-site/14-03-13-11-58-12}
    \end{subfigure}
  } \\[1ex]
  \makebox[\textwidth][c]{
    \begin{subfigure}{0.55\textwidth}
      \includegraphics[width=\textwidth]{sample-rfi-site/14-03-13-12-00-38}
    \end{subfigure}
    \begin{subfigure}{0.55\textwidth}
      \includegraphics[width=\textwidth]{sample-rfi-site/14-03-13-12-08-11}
    \end{subfigure}
    \begin{subfigure}{0.55\textwidth}
      \includegraphics[width=\textwidth]{sample-rfi-site/14-03-13-12-19-14}
    \end{subfigure}
  } \\[1ex]
  \makebox[\textwidth][c]{
    \begin{subfigure}{0.55\textwidth}
      \includegraphics[width=\textwidth]{sample-rfi-site/14-03-13-12-08-11}
    \end{subfigure}
    \begin{subfigure}{0.55\textwidth}
      \includegraphics[width=\textwidth]{sample-rfi-site/14-03-13-12-19-14}
    \end{subfigure}
    \begin{subfigure}{0.55\textwidth}
      \includegraphics[width=\textwidth]{sample-rfi-site/unknown}
    \end{subfigure}
  } \\[1ex]
  \makebox[\textwidth][c]{
    \begin{subfigure}{0.55\textwidth}
      \includegraphics[width=\textwidth]{sample-rfi-site/14-03-13-12-24-03}
    \end{subfigure}
    \begin{subfigure}{0.55\textwidth}
      \includegraphics[width=\textwidth]{sample-rfi-site/14-03-19-14-05-24}
    \end{subfigure}
    \begin{subfigure}{0.55\textwidth}
      \includegraphics[width=\textwidth]{sample-rfi-site/14-03-13-11-23-59.png}
    \end{subfigure}
  }
  \caption{Selection of impulses collected at MeerKAT site with UNKNOWN origin. Construction was taking place on site at the time. Plots are time domain. X-axis is time in microseconds. Y-Axis is ADC output number of 8-bit ADC}
  \label{fig:field-trials:unknown-time-domain}
  \end{figure}
\end{landscape}

\begin{landscape}
  \thispagestyle{empty}
  \begin{figure}
  \centering
  \makebox[\textwidth][c]{
    \begin{subfigure}{1.1\textwidth}
      \includegraphics[width=0.48\textwidth]{sample-rfi-site/bakkie-startup0}
      \includegraphics[width=0.48\textwidth]{sample-rfi-site/bakkie-startup1}
      \caption{Bakkie starting up}
    \end{subfigure}
    \begin{subfigure}{0.55\textwidth}
      \includegraphics[width=\textwidth]{sample-rfi-site/two-way-radio}
      \caption{Keying the 70 MHz two way radio}
    \end{subfigure}
  } \\[1ex]
  \makebox[\textwidth][c]{
    \begin{subfigure}{0.55\textwidth}
      \includegraphics[width=\textwidth]{sample-rfi-site/unplugging-charger}
      \caption{Unplugging laptop charger}
    \end{subfigure}
    \begin{subfigure}{1.1\textwidth}
      \includegraphics[width=0.48\textwidth]{sample-rfi-site/connecting-charger0}
      \includegraphics[width=0.48\textwidth]{sample-rfi-site/connecting-charger1}
      \caption{Plugging in laptop charger}
    \end{subfigure}
  } \\[1ex]
  \makebox[\textwidth][c]{
    \begin{subfigure}{1.65\textwidth}
      \centering
      \includegraphics[width=0.32\textwidth]{sample-rfi-site/refrigirator0}
      \includegraphics[width=0.32\textwidth]{sample-rfi-site/fridge1}
      \includegraphics[width=0.32\textwidth]{sample-rfi-site/fridge2}
      \caption{Refrigeration compressor turning on}
    \end{subfigure}
  }
  \caption{Selection of impulses collected at MeerKAT site with KNOWN origin. Plots are time domain. X-axis is time in microseconds. Y-Axis is ADC output number of 8-bit ADC}
  \label{fig:field-trials:known-time-domain}
  \end{figure}
\end{landscape}

\section{Second trial: Final DF system}
Once the full system that has been describe in the previous Chapters had been completed, the second set of field trials took place. This was done at the University of Cape Town (UCT) sports field where we could more freely generate RFI. The tests involved setting the system up in the center of the field and walking various RFI sources around the field in a circle and measuring how well they were tracked. This section first explains how the system was made fully portable, then it proceeds to show the setup and discuss how the measurements were taken, and finally it looks at the results for each signal source that was trialed.

\subsection{Portability}
\begin{figure}
  \centering
  \includegraphics[width=0.5\textwidth]{atx-psu}
  \caption{Mini-Box PicoPSU which plugs into the ROACH motherboard allowing it to run directly from a battery}
  \label{fig:field-trials:atx-psu}
\end{figure}
It was necessary to power the ROACH from a battery in order to allow it to be portable and taken out into the open field.
Initially the plan was to power it from an inverter running off of a battery, however in order to reduce switching noise emissions, a batter powered ATX power supply was used instead. The ATX PSU was a Mini-Box PicoPSU-80-WI-32V which runs directly from a \SI{12}{\volt} battery. 
It can output \SI{80}{\watt} which is more than enough to run the ROACH; testing in the lab showed that the ROACH pulled \SI{3.1}{\ampere} at \SI{12}{\volt} which is \SI{37}{\watt}.
To connect it, the traditional mains-powered ATX powersupply is disconnected from the motherboard and this module is plugged into the motherboard. 
This is shown in \cref{fig:field-trials:atx-psu}.

A ROYAL 1150K battery was used to power the ROACH and laptop in the field. 
This is a \SI{105}{\ampere\hour} deep cycle calcium battery.
As the ROACH draws \SI{3.1}{\ampere} at \SI{12}{\volt}, meaning a running time of \(\frac{105}{3.1} = \SI{34}{\hour}\).
It is advised to not run down below \SI{70}{\percent} to maintain the battery lifespan. Even so, that's more than \SI{20}{\hour} of runtime in the field, which is significantly longer than needed.


\subsection{Setup and Test Procedure}

\begin{figure}
  \centering
  \begin{subfigure}{0.8\textwidth}
    \includegraphics[angle=-90,width=\textwidth]{steve-with-setup-close-up}
  \end{subfigure}
  \caption{Setup focused on receiver and front end. On the left from top to bottom: antenna array, SMA cables into RF front end, SMA into ROACH (shiny box under the laptop). Roach running directly off of \SI{12}{\volt} battery in the red and black box. Blue inverter for charting laptop between tests. Grey box on the left is R\&S spectrum analyser for checking the power level coming out of the LNAs before connecting to ROACH ADCs.  On the right is a zoom in of the RF front end located in top of the tripod under the antenna array.}
  \label{fig:field-trials:setup-closeup}
\end{figure}

\begin{figure}
  \centering
  \begin{subfigure}[b]{0.70\textwidth}
    \centering
    \includegraphics[width=\textwidth]{steve-with-setup-far-away}
  \end{subfigure}
  \begin{subfigure}[b]{0.24\textwidth}
    \centering
    \includegraphics[width=\textwidth]{valon-on-a-stick}
  \end{subfigure}
  \caption{Photo showing how the transmitter was carried around, with both it and the receiving antennas at a height of approximately \SI{2}{\meter}. The PCB in the middle is the Valon synthesiser. The shield of the USB cable going down and the wire coming out of the SMA port pointing up act as a quarter-wave dipole.}
  \label{fig:field-trials-transmitter-on-stick}
\end{figure}

\begin{figure}
  \centering
  \includegraphics[width=0.80\textwidth]{gps-tracks-all-with-measurement-zoomed-out}
  \caption{Route taken according to GPS logger. The curve is actually a collection of timestamped lat/long pairs joined smoothly}
  \label{fig:field-trails:gps-tracks-all}
\end{figure}

These field trials would involve walking signal sources around the DF system. Calculations and simulations were done to model ground reflection multipath interference and error due to non-flat wave fronts, which showed that in order to achieve suitably low errors the radius which should be used when walking around the DF station should be \SI{35}{\metre} and the hight above ground of both the DF antennas and the transmitter should be \SI{2}{\meter}. 

The antennas and LNAs were attached to a tripod and set up in the center of the field. Initial power measurements were done using a spectrum analyser to measure environmental noise and set attenuators appropriately. The ROACH and laptop were set up under the tripod, with a shielded Ethernet cable connecting them. This setup is shown in \Cref{fig:field-trials:setup-closeup}. Various transmitters (each has its own subsection following this) were attached to a wood pole so that they could easily be carried when walking around the field. One of these transmitters is shown in \Cref{fig:field-trials-transmitter-on-stick}. A person carried the transmitter, walking slowly and also keeping a GPS logger on them. The GPS logger was a mobile phone which took a timestamped GPS reading every 1 second and wrote it to a CSV file on the phone. The plot of the route walked over the course of the measurements is shown in \Cref{fig:field-trails:gps-tracks-all}.

Before doing the trail for each transmitter, the spectrum analyser was used to figure out the transmit frequency, and then the DF system was configured to lock on to the strongest observed frequency in a small range (a few channels) around the peak of the transmitter.

After the field trials for each transmitter were complete, the readings from the GPS logger were converted to angle measurements by running each time/coordinate reading, along with the fixed coordinates of the DF station through a Python library that converted the coordinate pair into an angle. This time/angle reading from the GPS logger was then plotted on top of the time/angle reading from the DF system to compare how well it tracked. This plot as well as a short discussion of it follow for each of the transmitters.


\subsection{HAM radio}
The first source which was trialed was a portable HAM radio, a Yaesu VX-7R, transmitting at \SI{223}{\mega\hertz} at \SI{17}{\dBm}. The track for this source is shown in \Cref{fig:field-trials:ham-source} indicating that in general it was tracked very well. Midway though the trials the person carrying the HAM radio accidentally released the transmit button, causing the DF system to lock on to the closest strongest signal which was a harmonic of a TV station transmitter. However, this was noticed immediately by the DF station operator as the results were being displayed real-time on screen. As such the operator could tell the carrier to re-walk the last few tens of meters of path.

\begin{figure}
  \centering
  \begin{subfigure}[b]{0.77\textwidth}
    \centering
    \includegraphics[width=\textwidth]{ham-radio-1-track-222-224}
  \end{subfigure}
  \begin{subfigure}[b]{0.22\textwidth}
    \centering
    \includegraphics[width=\textwidth]{ham-source}
  \end{subfigure}
  \caption{DF results for HAM radio tracking. Red: GPS track of real angle. Blue and green: DF output for angle of frequency}
  \label{fig:field-trials:ham-source}
\end{figure}

\subsection{Raspberry PI}
\begin{figure}
  \centering
  \begin{subfigure}[b]{0.77\textwidth}
    \centering
    \includegraphics[width=\textwidth]{pi-narrow}
  \end{subfigure}
  \begin{subfigure}[b]{0.22\textwidth}
    \centering
    \includegraphics[width=\textwidth]{pi-source}
    \vspace{2em}
  \end{subfigure}
  \caption{DF results for Raspberry Pi tracking. Red: GPS track of real angle. Blue and green: DF system output.}
  \label{fig:field-trials:ham-source}
\end{figure}
The HAM radio transmitted quite a strong signal, and a different device was necessary to contrast performance under weak signal conditions. For this, a Raspberry Pi had an application called \lstinline{fm-transmitter} installed on it that allows the Pi to broadcast an FM radio station by driving one of its GPIO pins. The application allows the carrier can be well above the usual FM band. For this test, the carrier was set to \SI{241.32}{\mega\hertz}, close to the maximum that the Pi could reliably generate. It was set to transmit a silent sound file to produce a continuous tone. A quarter wavelength wire \SI{0.3}{\metre} long was connected to the pin as one half of a dipole and the shielding of the USB power cable used as the other half, producing a fairly well defined antenna with vertical polarisation.

With the GPIO pin toggling at such a high frequency, only about \SI{100}{\milli\volt} peak-to-peak voltage was present. This into the approximately \SI{75}{\ohm} of the half wave dipole means a EIRP of \SI{-18}{\dBm} and received power of \SI{-69}{\dBm}\footnote{\(P_r = P_t G_t G_r \left( \frac{\lambda}{4 \pi R} \right)^2 = \SI{16}{\micro\watt} \times 1 \times 1 \left( \frac{\SI{1.2}{\meter}}{4 \pi \times \SI{60}{\meter}} \right)^2 = \SI{119}{\pico\watt} = \SI{-69}{\dBm}\)}. Despite this much lower signal strength, the DF system was still able to track the Pi remarkably well, with an error only marginally higher than the HAM radio track.\\

\subsection{Valon Synthesiser}
\begin{figure}
  \centering
  \begin{subfigure}[b]{0.77\textwidth}
    \centering
    \includegraphics[width=\textwidth]{valon-track-narrow}
  \end{subfigure}
  \begin{subfigure}[b]{0.22\textwidth}
    \centering
    \includegraphics[width=\textwidth]{vaylon-source}
    \vspace{2em}
  \end{subfigure}
  \caption{DF results for Raspberry Pi tracking. Red: GPS track of real angle. Blue and green: DF system output.}
  \label{fig:field-trials:valon-source}
\end{figure}
\begin{figure}
  \centering
  \includegraphics[width=0.8\textwidth]{valong-spectrum}
  \caption{Valon spectrum}
  \label{fig:field-trails-valon-spectrum}
\end{figure}

The final narrow-band transmitter which was tracked was a Valon synthesiser, the same device used in the ROACH to generate a clock signal. The plot is shown in \Cref{fig:field-trials:valon-source}. It was configured to generate at a higher frequency, \SI{257}{\mega\hertz}. The result is similar to that of the HAM radio and Pi in that it was tracked very well. One interesting thing to note was that this frequency was right next to another fixed transmitter (TV, perhaps?) and while the device was being walked around the field, when it got close to being in line with the fixed transmitter its angle was "pulled" to point to the fixed transmitter. This illustrates one of the risks of having multiple transmitters in or close to each other's frequency channel \Cref{fig:field-trails-valon-spectrum}.

\subsection{Spark Generator}

\begin{figure}
  \centering
  \begin{subfigure}[b]{0.82\textwidth}
    \centering
    \includegraphics[width=\textwidth]{impulse-track-unfiltered}
  \end{subfigure}
  \begin{subfigure}[b]{0.17\textwidth}
    \centering
    \includegraphics[width=\textwidth]{lighter-source}
  \end{subfigure}
  \caption{Spark Source}
  \label{fig:field-trials:impulse-source}
\end{figure}

\begin{figure}
  \centering
  \begin{subfigure}[b]{0.8\textwidth}
    \centering
    \includegraphics[width=\textwidth]{impulse-freq-unfiltered}
  \end{subfigure}
  \begin{subfigure}[b]{0.8\textwidth}
    \centering
    \includegraphics[width=\textwidth]{impulse-freq-filtered}
  \end{subfigure}
  \caption{Spark Source}
  \label{fig:field-trials:impulse-source-freq-filtering}
\end{figure}

\begin{figure}
  \centering
  \begin{subfigure}[b]{0.8\textwidth}
    \centering
    \includegraphics[width=\textwidth]{impulse-time-untiltered}
  \end{subfigure}
  \begin{subfigure}[b]{0.8\textwidth}
    \centering
    \includegraphics[width=\textwidth]{impulse-time-filtered}
  \end{subfigure}
  \caption{Spark Source}
  \label{fig:field-trials:impulse-source-freq-filtering}
\end{figure}

\begin{figure}
  \centering
  \includegraphics[width=0.8\textwidth]{impulse-track-filtered}
  \caption{Valon spectrum}
  \label{fig:field-trails:impulse-source-final-track}
\end{figure}

\section{Summary}
This chapter has looked at the results obtained from two field trials. The first trial at the SKA reserve captured real impulses and showed that since there was no similarity between different types of impulsive signals, power detection should be used in time-domain capture. The second trial was of the complete system at the UCT sports field. The trial showed that the DF system performed well at tracking narrow band signals, even weak ones in the presence of other signals. The tracking of impulsive signals was shown to be acceptable after a cleaning up of the data which was necessitated by the environment being noisy.

