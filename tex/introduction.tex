\chapter{Introduction}
\section{Background}

\section{User Requirements}

A system to perform geolocation of RFI sources is to be designed.
\begin{enumerate}
  \item This system must operate in the context of the SKA site, implying the following:
  \begin{enumerate}
    \item In general, the RF environment is sparse
    \item The system must not interfere with the telescopes. The main implication here is that the emissions generated by the system are below a level detectable to the telescope. 
    \item  Elements exposed to the environment should be hardened against strong winds, high temperatures and sand.
  \end{enumerate}

  \item It should have the ability to detect the presence of radio emissions which fall within the band of interest of the SKA telescope; \SI{600}{\mega\hertz} MHz to \SI{1.7}{\giga\hertz}.
    
  \item The system should provide the following functionality to the user:
  \begin{enumerate}
    \item Display the user's location on an on-screen map.
    \item Plot direction vectors from the user's location to the source of the detected RFI
    \item It should be able to differentiate between different emissions, and provide information about the location of the source of the emissions to a user. 
    \item These should appear both as a graphical display of the RFI locations on a map, as well as provide the information in a manner accessible programmatically, such as through an API. 
  \end{enumerate}

  \item Some characteristics of the RFI which must be detected include:
  \begin{enumerate}
    \item Pulsed, broadband signals with pulses as short as 100 ns. Here, an impulse of \SI{1}{\joule} foobar.
    \item Continuous, narrow band signals
    \item Signal power as low as -40 dBm for pulsed signals, and -90 dBm for CW signals
    \item The probability of false alarm should be around 5\% 
    \item The probability of intercept of a 100 ns pulse should be less than 100\%
  \end{enumerate}

  \item The estimated direction should have RMS error of no more than 5 degrees.
    
  \item . 

  \item The system should output a signal descriptor word to an ethernet network for integration into a larger system.  This word should conform to the format of a well-known standard.

  \item As well as providing this data programmatically via an API, the system should also display the estimated location for RFI sources on an interactive map, such as google maps. This requires the ability for a user to provide input into the system and the system to display to a monitor. 

  \item Furthermore, independently of whether RFI sources have been detected or not, the system should also provide regular spectral dumps which have been averaged for over 10 seconds. This can be used by other systems for detecting RFI buried in the noise. Again, these should be provided over ethernet.

  \item The system should be compact enough that it could be mounted to a bakkie and driven around site for RFI hunting campaigns

  \item The system should not introduce any additional RFI to the site.
\end{enumerate}
