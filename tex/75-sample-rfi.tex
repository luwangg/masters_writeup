\section{First trial: Sample Impulsive RFI}

A few months after the start of the project, the system consisted of a simple two-element antenna array and a simple ROACH design. The ROACH streamed input from a single two-channel ADC, did threshold detection and stored a small snapshot of raw time domain ADC samples to FPGA Block RAM (BRAM). Computer code was at this point only able to pull raw samples, save them, and do in memory time domain cross-correlation. 

This system was taken to the SKA's MeerKAT site in the Karoo with the objective of getting some snapshots of RFI to see what characteristics the impulses had as well as to attempt impulsive RFI hunting. Two types of antenna were used: printed LPDA antennas and omni-directional conical antennas.
The two antenna types and the setup and operation are shown in \Cref{fig:field-trials:first-system-setup}. Simple power detection was done by FPGA firmware which was designed to detect when the received signal amplitude went above a threshold,\footnote{Threshold was set by sampling the ambient signal amplitude for a few minutes and defining the threshold two standard deviations above ambient.} and then to capturing the pulse to DRAM. Once captured, detection was paused and the computer notified of the pulse. The computer would then read out the samples and re-enable detection when finished. Examples of captured signals of unknown origin are shown in \Cref{fig:field-trials:unknown-time-domain} and signals of known origin in \Cref{fig:field-trials:known-time-domain}. From this, we can see that there are almost no characteristics that all of the impulsive signals had in common. They varied in terms of all key pulse classification parameters: pulse length, shape, frequency content and repetition pattern. This fact contributed to the decision to do power detection for the impulsive signals. Since it would not be practical to attempt any form of matched filtering when hunting pulses that are of unknown origin, all that can be done is to look for a burst of energy in the time domain.

Direction finding from the raw time domain signals was attempted on the computer by doing the time domain cross-correlation process described earlier. However, these attempts to hunt for RFI using the two-element system did not prove successful due to the \SI{180}{\degree} ambiguity inherent in a two-element design and due to calibration techniques not having yet been developed at that early stage of the project. Also, at that stage the system needed to be plugged in to run and hence was constrained to be close to a power source.

\afterpage{
  \thispagestyle{empty}
  \begin{figure}
    \vspace{-4em}
    \centering
    \begin{subfigure}{0.8\textwidth}
      \centering
      \includegraphics[width=\textwidth]{first-trip-yagi-antennas}
      \caption{Two element array of compact printed LPDA antennas.}
    \end{subfigure}
    \begin{subfigure}{0.9\textwidth}
      \centering
      \includegraphics[width=\textwidth]{first-trip-omni-antennas}
      \caption{Two element array of EM-6916 Omni-directional conical antennas.}
    \end{subfigure}
    \begin{subfigure}{0.9\textwidth}
      \centering
      \includegraphics[width=\textwidth]{first-trip-processing}
      \caption{Laptop linked via Ethernet to the ROACH, processing collected data in real time.}
    \end{subfigure}
    \caption{Setup for first field tests at the SKA's MeerKAT site in the Karoo. Early 2014.}
    \label{fig:field-trials:first-system-setup}
  \end{figure}
  \clearpage
}
%TODO: put the following back
%\skiptoevenpage
\begin{landscape}
  \thispagestyle{empty}
  \begin{figure}
  \centering
  \makebox[\textwidth][c]{
    \begin{subfigure}{0.55\textwidth}
      \includegraphics[width=\textwidth]{sample-rfi-site/14-03-13-11-23-59}
    \end{subfigure}
    \begin{subfigure}{0.55\textwidth}
      \includegraphics[width=\textwidth]{sample-rfi-site/14-03-13-11-56-07}
    \end{subfigure}
    \begin{subfigure}{0.55\textwidth}
      \includegraphics[width=\textwidth]{sample-rfi-site/14-03-13-11-58-12}
    \end{subfigure}
  } \\[1ex]
  \makebox[\textwidth][c]{
    \begin{subfigure}{0.55\textwidth}
      \includegraphics[width=\textwidth]{sample-rfi-site/14-03-13-12-00-38}
    \end{subfigure}
    \begin{subfigure}{0.55\textwidth}
      \includegraphics[width=\textwidth]{sample-rfi-site/14-03-13-12-08-11}
    \end{subfigure}
    \begin{subfigure}{0.55\textwidth}
      \includegraphics[width=\textwidth]{sample-rfi-site/14-03-13-12-19-14}
    \end{subfigure}
  } \\[1ex]
  \makebox[\textwidth][c]{
    \begin{subfigure}{0.55\textwidth}
      \includegraphics[width=\textwidth]{sample-rfi-site/14-03-13-12-08-11}
    \end{subfigure}
    \begin{subfigure}{0.55\textwidth}
      \includegraphics[width=\textwidth]{sample-rfi-site/14-03-13-12-19-14}
    \end{subfigure}
    \begin{subfigure}{0.55\textwidth}
      \includegraphics[width=\textwidth]{sample-rfi-site/unknown}
    \end{subfigure}
  } \\[1ex]
  \makebox[\textwidth][c]{
    \begin{subfigure}{0.55\textwidth}
      \includegraphics[width=\textwidth]{sample-rfi-site/14-03-13-12-24-03}
    \end{subfigure}
    \begin{subfigure}{0.55\textwidth}
      \includegraphics[width=\textwidth]{sample-rfi-site/14-03-19-14-05-24}
    \end{subfigure}
    \begin{subfigure}{0.55\textwidth}
      \includegraphics[width=\textwidth]{sample-rfi-site/14-03-13-11-23-59.png}
    \end{subfigure}
  }
  \caption{Selection of impulses collected at MeerKAT site with UNKNOWN origin. Construction was taking place on site at the time. Plots are time domain. X-axis is time in microseconds. Y-Axis is ADC output number of 8-bit ADC}
  \label{fig:field-trials:unknown-time-domain}
  \end{figure}
\end{landscape}

\begin{landscape}
  \thispagestyle{empty}
  \begin{figure}
  \centering
  \makebox[\textwidth][c]{
    \begin{subfigure}{1.1\textwidth}
      \includegraphics[width=0.48\textwidth]{sample-rfi-site/bakkie-startup0}
      \includegraphics[width=0.48\textwidth]{sample-rfi-site/bakkie-startup1}
      \caption{Bakkie starting up}
    \end{subfigure}
    \begin{subfigure}{0.55\textwidth}
      \includegraphics[width=\textwidth]{sample-rfi-site/two-way-radio}
      \caption{Keying the 70 MHz two way radio}
    \end{subfigure}
  } \\[1ex]
  \makebox[\textwidth][c]{
    \begin{subfigure}{0.55\textwidth}
      \includegraphics[width=\textwidth]{sample-rfi-site/unplugging-charger}
      \caption{Unplugging laptop charger}
    \end{subfigure}
    \begin{subfigure}{1.1\textwidth}
      \includegraphics[width=0.48\textwidth]{sample-rfi-site/connecting-charger0}
      \includegraphics[width=0.48\textwidth]{sample-rfi-site/connecting-charger1}
      \caption{Plugging in laptop charger}
    \end{subfigure}
  } \\[1ex]
  \makebox[\textwidth][c]{
    \begin{subfigure}{1.65\textwidth}
      \centering
      \includegraphics[width=0.32\textwidth]{sample-rfi-site/refrigirator0}
      \includegraphics[width=0.32\textwidth]{sample-rfi-site/fridge1}
      \includegraphics[width=0.32\textwidth]{sample-rfi-site/fridge2}
      \caption{Refrigeration compressor turning on}
    \end{subfigure}
  }
  \caption{Selection of impulses collected at MeerKAT site with KNOWN origin. Plots are time domain. X-axis is time in microseconds. Y-Axis is ADC output number of 8-bit ADC}
  \label{fig:field-trials:known-time-domain}
  \end{figure}
\end{landscape}
