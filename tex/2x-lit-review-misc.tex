\section{Geolocation}
Geolocation refers to the process of finding the absolute position of a target, often in terms of a coordinate system like latitude/longitude/elevation. This information is often more useful than only knowing the direction which an emitter lies in. However, it is shown that by having multiple DF stations, the process of triangulation may be used to geolocate a device from direction bearings \cite{poisel2012electronic}. 

This process is shown graphically in \cref{fig:lit-triangulation-from-df}, where multiple DF stations ($S_{1}$ through to $S_{M}$) are used to locate the x,y,z coordinates of the target $x_{T}$. Note that this geolocation process in the figure is for airborne DF systems searching for a ground based target. However, the system could easily be simplified to a purely terrestrial process.

The relevance of this note about geolocation to this work is that it is not necessary to attempt to design a system which can do geolocation natively. Rather, a DF system can be design which can later be duplicated in order to provide geolocation capabilities. 

\begin{figure}[p] 
  \centering
  \includegraphics[width=0.6\textwidth]{./img/lit_review/triangulation_from_df.pdf}
  \caption{Using triangulation from multiple DF stations to ascertain the geographic location of a target. Source: \cite{poisel2012electronic}}
  \label{fig:lit-triangulation-from-df}
\end{figure}

\section{Radiometer}
A radiometer is a device which is able to provide a very high accuracy approximation of noise power by averaging a large number of noise samples over a long time period. Due to the high accuracy measurements it can make, it is able to detect small changes in received power. 
It does this by achieving a high sensitivity. Sensitivity is the measure of how weak a signal and instrument is able to detect.

We will define power in terms of a matched load heated to a certain temperature observed over a certain bandwidth.
\begin{equation}
  P = kTB
\end{equation}
where \(k \approxeq \SI{1.38e-23}{\joule\per\kelvin}\), the Boltzmann constant. 

System temperature is a combination of noise powers from atmospheric emissions, the warm earth, the cosmic microwave background, receiver noise figure, losses in the RF chain and others. These noise sources mask the RFI signal which we are trying to locate. While typically an RFI signal below the noise floor would not be an issue, in the case of a radio astronomy reserver that signal will in all likelihood still be a problem because the noise floor of the MeerKAT array is so much lower than the RFI DF instrument. In essence, a signal which will be very loud to the telescope may be difficult for the DF system to even detect. Effort will hence need to go into ensuring that this instrument being designed will be able to see signals below its own noise floor. 

As the noise at an antenna is a combination of a multitude of noise sources, the central limit theorem states that output of the antenna will be approximately a normal (Gaussian) distribution. 

The noise temperature is defined as the noise power per unit bandwidth over the Boltzmann constant:
\begin{equation}
  T_N = \frac{P_v}{k}
\end{equation}

Assuming the power of a noise source remains constant, a single sample of the noise source has an RMS error of approximately \(\sqrt{2}T_{sys}\). However, by integrating the noise power from a certain bandwidth \(\Delta B\) over some integration time \(\tau\) the error can be significantly reduced to
\begin{equation}
  \sigma_{T} = \frac{\sqrt{2}T_{sys}}{\sqrt{2B\tau}} = \frac{T_{sys}}{\sqrt{B \tau}}
\end{equation}
Where \(\sigma_{T}\) is the RMS error in the measurement of the noise temperature, \(T\), and \(T_{sys}\) is the actual system noise temperature. Note that \(2B\tau\) is the number of samples acquired

This process used in a radiometer of acertaining a high accuracy approximation of a received signal by integrating the signal over some observation period can also be used in the context of a direction finding system. For this application, a long observation period may be used to extract a weak signal which is burried in noise so that the weak signal may be processed.

\begin{equation}
  \frac{S}{N} = \frac{S}{N}\sqrt{B\tau}
\end{equation}

Tsys = Tsky + Trx where Tsky is thestuff above and Trx is Johnson noise from electronic components. Tsky can't do anything about. Trx lowered by cooling components. 
The Trx is as a result of the Johnson-Nyquist noise. 

Extension to interferometry:

\(SEFD / (N(N-1))/2 \tau 2B)\)

Sources: \url{https://casper.berkeley.edu/astrobaki/index.php/Radiometer_Equation}
\url{https://casper.berkeley.edu/astrobaki/index.php/Radiometer_Equation_Applied_to_Telescopes}
\url{https://www.cv.nrao.edu/course/astr534/Radiometers.html}

\section{Suspect I should delete}
\subsection{Signals}
The \(M \times 1\) steering matrix is
\begin{equation}
  \vec{a}_k(\vec{\theta}_k) = 
  \begin{bmatrix}
    e^{-j\omega_c \tau_1(\phi_k)} \\
    e^{-j\omega_c \tau_2(\phi_k)} \\
    ... \\
    e^{-j\omega_c \tau_M(\phi_k)} \\
  \end{bmatrix}
\end{equation}

Let there be $N$ individual signal sources, where $\vec{s}(t)$ represents the resultant signal, being
\begin{equation}
\vec{s}(t) = \begin{bmatrix} s_{1}(t) & s_{2}(t) & s_3(t) & ... & s_N(t) \end{bmatrix}
\end{equation}
As discussed by \cite{krim1996two}:

We are interested in extracting a parameter of a signal. Specifically, the parameter of interest is \gls{aoa}. This is what sensor array signal processing does.

We model the E-field of a narrow-band signal by:
\begin{equation}
  E(\vec{r}, t) = s(t) \exp \left\{ j(\omega t - \vec{r}\, \TRANSPOSE \vec{k}) \right\}
\end{equation}

Where:
\begin{itemize}
  \item \(s(t)\) is the slow (compared to the carrier) modulating signal with bandwidth \(B\). This would be a constant for a signal that is a pure tone.
  \item \(\omega\) is the carrier frequency
  \item \( \vec{r} \) is the radius vector, of form \( \begin{bmatrix} x, y, z, \end{bmatrix} \).
  \item \(\vec{k} = \alpha\omega\) which is the wave-vector where \(\alpha = \frac{1}{c}\) pointing in the direction of propagation. Note that the magnitude of the wave-vector is known as the wave-number: \(\lvert \vec{k} \rvert = k = \frac{\omega}{c} = \frac{2\pi}{\lambda}\). This implies: \(\vec{k} = k(\cos\theta \sin\theta)\TRANSPOSE\) where \(\theta\) is the angle of the incident wave.
\end{itemize}

If we have a receiver with a radius vector \(\vec{r_r} = \begin{bmatrix} x_r, y_r \end{bmatrix}\TRANSPOSE\)

Note that as per the narrowband assumption is is assumed that the array aperture be much less than the inverse relative bandwidth \((f/B)\)

It is shown that the output of an \(L\)-element array a \(L\)-dimensional vector of the steering vector scalar multiplied by the incident signal, given by
\begin{equation}
  \vec{x}(t) = \vec{a}(\theta).s(t)
\end{equation}

Here, \(\vec{a}(\theta) = \begin{bmatrix} a_1(\theta), a_2(\theta), ... , a_L(\theta) \end{bmatrix}\TRANSPOSE\) which is the steering vector.

  Furthermore, it is shown that the principle of superposition applies. If there are \(M\) incident signals they are simply summed together:
\begin{equation}
  \vec{x}(t) = \sum_{m=1}^{M} \vec{a}(\theta_m)\vec{s_m}(t)
\end{equation}

This can be re-written in a more compact form (now adding noise to the model):
\begin{equation}
  \vec{x}(t) = \mathbf{A}(\vec{\theta})\vec{s}(t) + \vec{n}(t)
\end{equation}

Where:
\begin{itemize} 
  \item we have re-written \(\sum_{m=1}^{M}\vec{a}(\theta_m)\) as a matrix of steering vectors
\begin{equation} 
  \mathbf{A}(\vec{\theta}) = \begin{bmatrix} \vec{a}(\theta_1), \vec{a}(\theta_2), ..., \vec{a}(\theta_M) \end{bmatrix} \\
\end{equation}
  \item we have re-written  \(\sum_{m=1}^{M}s_m(t)\) as a vector:
\begin{equation}
  \vec{s}(t) = \begin{bmatrix} s_t(t) \\ .. \\ s_M(t) \end{bmatrix}
\end{equation}
\end{itemize}

As discussed by \cite{sleiman2000antenna} \cite{karimi1996manifold} \cite{dacos1995estimating}. 

The antenna array manifold is said to be useful for direction finding systems, as signal subspace techniques such as MuSIC rely on searching for the best \(\vec{a}(p)\) for the detected signal \cite{karimi1996manifold}. 

It is shown by \cite{dacos1995estimating} that the output of an array of N sensors receiving M signals in the presence of noise is
\begin{equation}
\vec{x}(t) = \mtx{A}(\vec{p})\vec{m}(t) + \vec{n}(t)
\end{equation}
Where \(\vec{x}(t)\) is the N-dimensional output of the array, \(\vec{m}(t)\) is the M-dimensional set of signals received by the array, and \(\mtx{A}(\vec{p})\) is a \(N \times M\) matrix of source position vectors (SPV). 
A given SPV, \(\vec{a}(p_i)\), shows how the array responds to a source at location \(p_i\), where \(p_i\) is often an azimuth and elevation pair: \(p_i = (\theta_i, \phi_i)\).

For the case of a terrestrial-only system (which this project will be concerned with), \(\phi\) can be set to 0, meaning that \(p_i = \theta_i\), the azimuth angle of source \(i\), typically in the range \([0, 2\pi]\). Here, a SPV can be simplified to \(\vec{a}(\theta_i)\).

It is shown that if the \(N\) antennas are positioned symmetrically, the antenna array manifold is reduced from complex space \(\mathbf{C}^N\) to real space \(\mathbf{R}^N\) \cite{dacos1995estimating}.
