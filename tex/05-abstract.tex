\chapter{Abstract}

For radio astronomy telescopes to be able to perform observations of weak signals from space, they need to operate in a radio-quiet environment. Any Radio Frequency Interference (RFI) will interfere with the ability of the telescope to collect data. With the proliferation of electrical and electronic devices, RFI management is one of the major challenges facing radio astronomy reserves. 

This thesis details the design, construction and testing of a system which is able to find the direction which a source of interference is coming from. User requirements for the system are captured, and one of the key requirement of the system is the ability to direction-find two classes of RFI: weak narrow-band continuous signals, and strong impulsive signals. Both of these classes of signals pose problems for radio telescopes. The primary focus of the thesis is implementing the algorithms to direction find those signals, and to evaluate whether the algorithms perform as expected on real RFI sources in the field.

An analysis of various prior direction finding techniques is done from the existing literature to select the most suitable technique for this system. A combination of phase interferometry and time difference of arrival are selected, due to their suitability for the classes of signals, the operating environment and the hardware that will be used. Simulations done showing how the system should operate and to highlight potential challenges. A key challenge is around phase ambiguity, and special attention is paid to mitigating this.

After design and simulation, a full system is implemented containing a number of sub systems linked together. A four element deformed circular antenna array and RF front end pick up signals from the environment. These signals are digitised together in phase by fast ADCs. The output of the ADCs goes into an FPGA on a ROACH board which does high speed DSP including Fourier transforms, spectrum cross correlations, accumulations, power detections and time domain capturing. The output of the DPS done on the FGPA is received by a computer running a Python application which performs the final angle of arrival calculations in real time. The application has a mathematical model of the antenna array which it combines with the received baseline time difference or phase shift measurements to ascertain the direction of the signal source.

When designing the building the system, emphasis is put in making it flexible and reconfigurable, allowing it to be used with arbitrary array configurations or frequency ranges.

The system is first put together and tested in the lab using signal generators, noise sources and impulse generators. These signals are fed into the ROACH to simulate an RF environment and hence ensure that the design is working as expected. Next, the system is made portable and taken out for field trials. The field trials demonstrate that the system is able to provide accurate tracks for a number of different RFI sources, both impulsive and narrowband. It is able to maintain a track over the full \SI{360}{\degree} field of view as required.
