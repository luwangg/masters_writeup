\chapter{Abstract}

For radio astronomy telescopes to be able to perform observations of weak signals from space, they need to operate in a radio-quiet environment. With the proliferation of electrical and electronic devices, RFI management is one of the major challenges facing radio astronomy reserves. 

This thesis details the design, construction and testing of a proof-of-concept system which is able to find the direction which a source of interference is coming from. The requirement of the system is the ability to direction-find two classes of RFI: weak narrow-band continuous signals, and strong impulsive signals. The primary focus of the thesis is evaluating whether the algorithms and implementation developed here are able to track real RFI sources in the field.

User requirements for the system are captured, and an analysis of direction finding techniques is done to select the most suitable technique for this systems. A combination of phase interferometry and time difference of arrival are selected, and simulations done showing how the system should operate and to highlight potential challenges.

After design and simulation, a full system is implemented containing a number of sub systems linked together. A four element deformed circular antenna array and RF front end pick up signals from the environment. These signals are digitised together in phase by fast ADCs. The output of the ADCs goes into a Virtex 5 FPGA on a ROACH board which does high speed DSP including FFTs, spectrum cross correlations, accumulations, power detections and time domain capturing. The output of the DPS done on the FGPA is routed to a computer running a Python application which performs the final angle of arrival calculations in real time. The application has a mathematical model of the antenna array which it combines with the received baseline time difference or phase shift measurements to ascertain the direction of the signal source.

The system is first put together in the lab using signal generators, noise sources and impulse generators being fed into the ROACH board to simulate an RF environment and hence ensure that the design is working as expected. Next, the system is made portable and taken out for field trials. The field trials demonstrated that the system was able to provide accurate tracks for a number of different RFI sources, both impulsive and narrowband. It was able to maintain a track over the full \SI{360}{\degree} field of view.
