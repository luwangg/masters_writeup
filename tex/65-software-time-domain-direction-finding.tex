\section{Time Domain Direction Finding}

As mentioned in Chapter 3, the algorithm behind time domain DF is remarkably similar to frequency domain DF. The main difference is that the comparison method uses baseline time delays rather than phase shifts. 

One important difference is that where the frequency domain cross-correlations are calculated using DSP on the FPGA, the time domain cross-correlations need to be calculated on the computer. This is not a hard requirement, but the implementation of the FPGA design did not cater for time domain cross-correlations which can be a lot trickier to do in hardware since the length of the pulse is dynamic.

The time domain cross-correlation is implemented, as defined, by multiplying each point of one antenna's signal \(\vec{a}\) by the corresponding points of a shifted version of another antenna's signal \(\vec{b}\) for some shift \(k\), producing the correlation \(c_{ab}(k)\):
\begin{equation} \label{eq:software:time-domain-cross-correlation}
  c_{ab}(k) = \sum_{n = 0}^{N-1} a_{n} b_{n+k}
\end{equation}

The range of values for \(k\) will be picked to span an interval a bit larger than the propagation time of a signal over the whole array. For the array being used here this will correspond to \(k\in [-100, 100]\), \(k\in \mathbb{Z}\)\footnote{This is about 10 times more than is necessary, but seeing as the correlation output will be upsampled the signal should not be short, or the frequency resolution available in the upsampling process will be limited, leading to an inaccurate output.}. To make this possible, \(\vec{b}\) is zero-padded with 100 zeros on each end. Hence, in \Cref{eq:software:time-domain-cross-correlation} the value of \(N\) is the number of points in the shorter, non-padded vector \(\vec{a}\).

Once an array of \(c\) values for various \(k\) values has been generated, the \(c\) array is upsampled using the Fourier method discussed in Chapter 3.  Upsampling of large signals can consume lots of CPU cycles and slow down the DF system. However by upsampling after cross-correlation we take advantage of the property that upsampling the raw time domain signals has the same effect as upsampling the output of the time domain cross-correlation. While impulses of even a few microseconds will be tens of thousands of ADC samples (which can be expensive to upsample), the time domain cross-correlation output will be only 200 samples as it's limited to the correlation interval. 

Once the upsampling is complete, calibration factors are applied to \(c\), then the maximum value of the upsampled calibrated \(c\) is found and the corresponding time shift noted. This is done for the snapshot of each combination of pairs of antennas, \(a\) and \(b\). The result is a map of baselines to observed time differences seen by that baseline.

The same algorithm as the frequency domain DF is then used to find the angle where the simulated baseline time delays most closely match the observed ones. That best-matching angle is then chosen as the angle or arrival.



