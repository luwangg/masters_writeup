\chapter{Antenna Array and RF Front End}
\label{ch:rf-front-end}
\setsvg{svgpath=./img/rf-front-end/}
\graphicspath{{./img/rf-front-end/}}

This chapter details the design and characterisation of the antennas and RF front end; the first sub-system in the direction finder. The RF front end consists of low noise amplifiers (LNA), lower pass filters, and cabling. As discussed in the previous chapters, the aspects of the array which needs to be optimised for are:
\begin{itemize}
  \item Minimize the phase difference the RF path for each antenna. The DF algorithm is based on phase/time difference measurements and hence path mismatches will result in errors. Some phase difference can be calibrated out but the calibration routine will be most robust when the initial error is low.
  \item The spacing of the elements in the circular array needs to trade off between too large an element spacing and too small a spacing. Too large will result in phase ambiguity, to small will result in coupling between the elements as well as difficulty in measure time difference. The compromise is to make the array as large as possible before phase ambiguity becomes a problem.
\end{itemize}

The ADCs which were available for use in this project influenced the antenna and operating frequency band choice. They're explored more in Chapter 5 but will be briefly noted here as justification for the decisions taken here. The ADCs were two dual-input cards which sample two inputs in phase. They were clocked up to \SI{800}{\mega\hertz}. This implies 4 simultaneous inputs with a Nyquist frequency of \SI{400}{\mega\hertz}.\\

The Chapter will first discuss the design and construction of the antenna array, then move on to the RF front end. Finally, characterisation of the amplitude and phase performance of the RF front end will then be done. 


\section{Antenna Array}

As shown in the simulations, it's preferable to use an odd number of elements in the circular array as the phase ambiguity is reduced when there are no lines of symmetry. However, there is a constraint in the hardware that's being used for this project: there are a total of four simultaneous ADC inputs. Rather than going down to three elements, all four inputs will be used but the circular array will be deformed in order to remove the lines of symmetry. The amount of deformation will be trade off between deviating too far from a circle and hence having performance that is not the same in all directions, or being too close to a circle and hence suffering from ambiguity. This project does not go into depth on array deformation strategies; the array design is proof of concept and the deformation we can do is limited by the mounting hardware. A few array geometries were tried and the one that performed best in ambiguity simulation.

\begin{figure}
  \centering
  \includegraphics[width=0.4\textwidth]{antenna-array-on-roof}
  \includegraphics[width=0.59\textwidth]{antenna-s11}
  \caption{Array of four FD-250 folded dipoles being measured and the result of the S11 measurements showing acceptable performance between \SI{200}{\mega\hertz} and \SI{300}{\mega\hertz}.}
  \label{fig:rf-front-end:antenna-array-s11}
\end{figure}

Four FD250 folded dipole antennas with a center frequency of \SI{250}{\mega\hertz} were sourced. This centre frequency was picked as it's around the middle of the \SI{400}{\mega\hertz} ADC usable bandwidth. The antennas were mounted in a deformed circle. Measurements were taken and acceptable S11 performance in the band of interest was observed; see \Cref{fig:rf-front-end:antenna-array-s11}.

In order to be able model the array in simulations and field trials, it was necessary to get coordinates for each element. A Python program was written which takes is input measurements of all of the baselines and does a least square errors calculation to estimate the coordinates of each element. A top down view of the real array as well as a plot of the elements produced by the script is show in \Cref{fig:rf-front-end:array-coordinates}.
Thoughts on what could go here:
- show multi-colour vis vs df error graph
- show individual fixed source plots
- show fixed freq 4 vs 4 deformed plots
- link to appendix for individually samples plots.

%TODO: put the following back
\begin{figure}
  \includegraphics[width=0.4\textwidth]{array-top-view}
  \includegraphics[width=0.59\textwidth]{array-coordinates}
  \caption{Top view with coordinates}
  \label{fig:rf-front-end:array-coordinates}
\end{figure}

\begin{figure}
  \centering
  \begin{subfigure}{\textwidth}
    \centering
    \includegraphics[width=0.90\textwidth, clip=true, trim = 0 14 50 0]{4}
    % left, bottom, right, top
  \end{subfigure}
  \begin{subfigure}{\textwidth}
    \centering
    \includegraphics[width=0.90\textwidth, clip=true, trim = 0 14 50 0]{4-deformed}
  \end{subfigure}
  \caption{Ambiguity plots for various antenna array sizes with reference signal arriving at \SI{0}{\degree} showing improved ambiguity after array deformation.}
\end{figure}

\begin{figure}
  \centering
  \includegraphics[width=0.7\textwidth, clip=true, trim = 0 0 60 0]{4-element-circular-ambiguity-vs-phi}\\2em
  \includegraphics[width=0.7\textwidth, clip = true, trim = 0 0 0 0]{4-element-deformed-ambiguity-vs-phi}
  % left, bottom, right, top
  \caption{Some caption here}
\end{figure}

RMS phase error vs RMS angular error. MOVE THIS TO FRONT-END SECTION.
\begin{figure}
  \centering
  \includegraphics[width=0.9\textwidth]{visibility-error-vs-df-error}
  \caption{This is a graphic showing some stuff}
\end{figure}

Again, simulation:
SNR vs number of correct correlation peaks.
Table: 
For an SNR: Do 1000 different cross correlations.RMS error in number of samples.
Do SNR: 10, 5, 2, 1, 0.5, 0.2, 0.1

\section{RF Front End}

\begin{figure}
  \centering
  \begin{subfigure}{\textwidth}
    \centering
    \includesvg[width=\textwidth]{fr-front-end-circuit-sch}
    \caption{Schematic of the RF front end circuitry.}
  \end{subfigure}\\[1em]
  \begin{subfigure}{\textwidth}
    \centering
    \includegraphics[width=0.6\textwidth]{rf-front-end-charging}
    \caption{Board containing amplifiers and filters in the corner, batteries velcroed down on the sides, and regulators and connectors on the veroboard in the middle.}
  \end{subfigure}
  \begin{subfigure}{\textwidth}
    \centering
    \includegraphics[width=0.6\textwidth]{lpf-in-holder}
    \caption{Close up of a low pass filter connected to an amplifier, held down to make it more rugged in the field.}
  \end{subfigure}
  \caption{RF front end design and construction.}
  \label{fig:rf-front-end:circuit-board}
\end{figure}

The RF front end is responsible for filtering and amplifying the signals, being the bridge between the antennas and the ADC. As discussed in Chapter 3, it is necessary for an interferometry system to have its RF front end components phase and amplitude response approximately matched.
The ADCs have \SI{400}{\mega\hertz} of usable bandwidth and the antennas have a centre frequency of \SI{250}{\mega\hertz}. Hence, the filtering should cut off before \SI{400}{\mega\hertz} and the amplification should occur around \SI{250}{\mega\hertz}. 

Suitable parts were purchased. The LNAs which are used are the ZLF-500HLN from MiniCircuits. This part operates from \SI{10}{\mega\hertz} to \SI{500}{\mega\hertz} which is ideal for the application. The gain is approximately 21 dB across this band according to the datasheet, drawing up to \SI{110}{\milli\ampere}.  The low pass filters (LPF) which were purchased are VLF-225+ parts from MiniCircuits having a \SI{3}{\decibel} point at \SI{350}{\mega\hertz}, which defines the usable bandwidth of the RF front end.

The RF front end needs to be able to be taken out into the field and used while running on batteries.
Two ZIPPY Compact \SI{1000}{\milli\ampere\hour} 3S 25C battery packs and 4 LM7815 regulators were acquired. 
Each amplifier has its own regulator to try to reduce any electrical coupling between the regulators through their power rails.
The battery packs output \SI{12.5}{\volt} when fully charged meaning a combined input to the \SI{15}{\volt} regulators of \SI{25}{\volt}. Hence the regulators are dropping \SI{10}{\volt} at \SI{100}{\milli\ampere} or \SI{1}{\watt}.
The LM7815s can handle this dissipation provided a heat sink is connected, which was done.
This power distribution circuitry was put onto veroboard.

All of the amplifiers and filters and the power distribution circuit were mounted to a wooden board.
Care was taken to make sure that the low pass filters were securely and firmly attached to the board as there is a risk that the SMA connector would snap off if they they were bumped. 
The circuit diagram for this board and the resulting hardware implementation for this RF front end are are shown in \Cref{fig:rf-front-end:circuit-board}.

Finally, the cables need to be carefully matched as well. Provided the cables are of the same type, their velocity factors will be equal and it is hence only necessary to ensure that they are of the same length. The LPFs and LNAs however may have phase mismatches due to manufacturing tolerances. 

In order to check whether the LNAs provide the expected gain and to check whether the phase matching of the RF front end is good, the whole subsystem of cables, amplifiers and filters were connected to a network analyser and S21 measurements taken. The results are shown in \Cref{fig:rf-front-end:vna-measurements}. It can be seen that the amplitude drops off rapidly at around \SI{400}{\mega\hertz} providing necessary anti aliasing. At \SI{350}{\mega\hertz} the difference between the in band signal and the aliased signal is almost \SI{-30}{\decibel} which is sufficient isolation to consider the system usable up to \SI{350}{\mega\hertz}. The phase matching is good, within a degree or two over the whole band of operation.

\begin{figure}
  \centering
  \begin{subfigure}{\textwidth}
    \centering
    \includegraphics[width=0.6\textwidth]{zfl500-10-3000}
    \caption{Gain from 0 to \SI{3}{\giga\hertz} in dB. Antialisaing is sufficient. The iADC is only receptive to signals up to \SI{3}{\giga\hertz} so it's only necessary to look at filtering up to there.}
  \end{subfigure}\\[1em]
  \begin{subfigure}{\textwidth}
    \centering
    \includegraphics[width=0.6\textwidth]{zfl500-10_500-mag-db}
    %\caption{Gain from 0 to \SI{500}{\mega\hertz} in dB. Useable seems to be up to \SI{350}{\mega\hertz}}
    \caption{VNA measurements of S21 of RF chains showing antialising properties}
  \end{subfigure}
  \begin{subfigure}{\textwidth}
    \centering
    \includegraphics[width=0.6\textwidth]{zfl500-10-500-phase}
    \caption{Phase from 0 to \SI{500}{\mega\hertz} showing very good phase matching over operating frequency range below \SI{350}{\mega\hertz}}
  \end{subfigure}
  \caption{S21 measurements of RF front end subsystem.}
  \label{fig:rf-front-end:vna-measurements}
\end{figure}


\section{Summary}
This Chapter has detailed the design, construction and measurement of the antenna array and RF front end subsystem. It's shown how a 4 element array can be deformed from a square to produce ambiguity performance that's not far off of what's achieved by a 5 element array. Code was written to be able to introduce a coordinate system to elements from baseline distance measurements. 

The RF front end was built to provide some amplification in the band of interest and attenuation of out-of-band aliased signals. Mounting the circuitry on a board and making it battery powered will allow it to be taken into the field easily.

Measurements of the performance of the RF front end show it should work as expected.
