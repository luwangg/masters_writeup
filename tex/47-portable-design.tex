\section{Portable Configuration}
The RF front end needs to be able to be taken out into the field and used while running on batteries.
Two ZIPPY Compact \SI{1000}{\milli\ampere\hour} 3S 25C battery packs and 4 LM7815 regulators were acquired. 
Each amplifier has its own regulator to try to reduce any electrical coupling between the regulators through their power rails.
The battery packs output \SI{12.5}{\volt} when fully charged meaning a combined input to the \SI{15}{\volt} regulators of \SI{25}{\volt}. Hence the regulators are dropping \SI{10}{\volt} at \SI{100}{\milli\ampere} or \SI{1}{\watt}.
The LM7815s can handle this dissipation provided a heat sink is connected, which was done.
This power distribution circuitry was put onto veroboard.

All of the amplifiers and filters and the power distrubution board were mounted to a wooden board.
Care was taken to make sure that the low pass filters were securely and firmly attached to the board as there is a risk that the SMA connector would snap off if they they were bumped. 
The circuit diagram for this board and the resulting hardware implementation for this RF front end are are shown in figure ??.


\begin{figure}
  \centering
  \begin{subfigure}{\textwidth}
    \centering
    \includesvg[width=\textwidth]{fr-front-end-circuit-sch}
    \caption{Schematic}
  \end{subfigure}\\[1em]
  \begin{subfigure}{\textwidth}
    \centering
    \includegraphics[width=0.6\textwidth]{rf-front-end-charging}
    \caption{The thing}
  \end{subfigure}
  \begin{subfigure}{\textwidth}
    \centering
    \includegraphics[width=0.6\textwidth]{lpf-in-holder}
    \caption{LPF in holder}
  \end{subfigure}
  \caption{RF Front end design and construction}
\end{figure}
