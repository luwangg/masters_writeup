\chapter{Field Trials}
\label{ch:field-trials}

\section{Power supply}
It was necessary to power the ROACH from a battery in order to allow it to be portable and taken out into the open field.
Initially the plan was to power it from an inverter running off of a battery. Here, discuss why the inverter may be too noisy. Can this be shown from reverb chamber measurements?

Instead, an ATX power supply which runs directly from a \SI{12}{\volt} battery was made available from the SKA equipment. The power supply is made by Mini-Box and its part number is PicoPSU-80-WI-32V. This can output \SI{80}{\watt} which is enough to run the ROACH. To connect it, the traditional mains-powered ATX powersupply is disconnected from the motherboard and this module is plugged into the motherboard. This is shown in (insert figure here!).

Furthermore, a ROYAL 1150K battery was supplied. This is a \SI{105}{\ampere\hour} deep cycle calcium battery.
As the ROACH draws X amps at \SI{12}{\volt}, this implies that we are discharging the battery at \(\frac{105}{X} = \frac{C}{Y}\).
It is advised to not run down below \SI{70}{\percent} to maintain the battery lifespan. As shown by (cite battery charging), this implies that the voltage should not drop to below \SI{12.1}{\volt} when discharging at \(\frac{C}{Y}\)

Testing in the lab showed that the ROACH pulled \SI{3.1}{\ampere} at \SI{12}{\volt} which is \SI{37}{\watt}. This can easily be handled by the \SI{80}{\watt} ATX power supply. 
Testing by running the ROACH from the battery overnight. The battery started at \SI{12.6}{\volt} and had dropped to \SI{11.9}{\volt} \SI{15}{\hour} later. The purpose of this test was not to provide a comprehensive report of the capacity of the battery or the requirements of the ROACH, but simply to show that the system will easily be able to run for a few hours in the field during field trials.

\section{Signal Source}
The signal source used is a portable HAM radio, lent by Jason Manley of the SKA. 
It's rated output power is \SI{5}{\watt}. Free space path loss equation will be calculated to get an approximation of the power going into the LNAs and the power going into the ADCs. At \SI{2500}{\mega\hertz}, \(\lambda = \SI{1.2}{\meter}\). The maximum gain of the FD-250 folded dipole is approximately \SI{0}{\dBi}.

\begin{align}
  P_r &= P_t G_t G_r \left( \frac{\lambda}{4 \pi R} \right)^2 \\
      &= \SI{5}{\watt} \times 1 \times 1 \left( \frac{\SI{1.2}{\meter}}{4 \pi \times \SI{60}{\meter}} \right)^2 \\
      &= \SI{0.0000127}{\watt} \\
      &= \SI{0.0127}{\milli\watt} \\
     &= \SI{-19}{\dBm}
\end{align}

After the \SI{20}{\decibel} gain of the ZFL-500HLN LNAs, the power into the iADCs is approximately \SI{1}{\dBm}. Seeing as the ADC's full scale range is \SI{0}{\dBm}, a \SI{10}{\decibel} attenuation will be inserted after each LNA to bring the input signal down to a safe level.
