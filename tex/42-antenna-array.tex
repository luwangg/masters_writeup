\section{Antenna Array}

As shown in the simulations in Chapter 3, it's preferable to use an odd number of elements in the circular array as the ambiguity is reduced when there are no lines of symmetry. However, as mentioned there's a constraint in the hardware that's being used for this project in that only four simultaneous inputs can be sampled. Options are hence to:
\begin{enumerate}
  \item go down to three elements to have an odd number, leaving one unused ADC input.
  \item use a five element array and build additional switching logic to selectively route four inputs to the ADCs
  \item use a four element array, sampling all elements and attempt to find a way to mitigate the ambiguity introduced by the symmetry of a four element array.
\end{enumerate}
    
Option 1 was ruled out as leaving an unused ADC input is wasting potential ambiguity-resolving input. The more input you have the better ambiguity can be resolved and the higher the accuracy of the measurement. 

Option 2 was ruled out for the complexity of building antenna switching circuitry and trying to dynamically route different inputs to the digitisers. 

As such, the approach taken is to construct a four element array using all four inputs. However, the circular array will be \emph{deformed} in order to remove the lines of symmetry. The amount of deformation will be trade off between deviating too far from a circle and hence having performance that is not the same in all directions, or being too close to a circle and hence suffering from ambiguity. This project does not go into depth on array deformation strategies; the array design is proof of concept and the deformation we can do is limited by the mounting hardware.\\

\begin{figure}
  \centering
  \includegraphics[width=0.4\textwidth]{antenna-array-on-roof}
  \includegraphics[width=0.59\textwidth]{antenna-s11}
  \caption{Array of four FD-250 folded dipoles being measured and the result of the S11 measurements showing acceptable performance between \SI{200}{\mega\hertz} and \SI{300}{\mega\hertz}.}
  \label{fig:rf-front-end:antenna-array-s11}
\end{figure}

Four FD250 folded dipole antennas with a center frequency of \SI{250}{\mega\hertz} were purchased. This centre frequency was picked as it's around the middle of the \SI{400}{\mega\hertz} ADC usable bandwidth. Measurements were taken and acceptable S11 performance in the band of interest was observed as shown in \Cref{fig:rf-front-end:antenna-array-s11}.
The antennas were mounted in a deformed circle. Deforming involved shortening one of the mounting booms, lengthening another one, and rotating a third boom by around \SI{20}{\degree}. 

In order to be able model the array in simulations and field trials, it was necessary to get coordinates for each element. A Python program\footnote{\url{https://github.com/jgowans/array-element-coordinate-calculator}} was written which takes as input measurements of all of the baselines and does a least square errors calculation to estimate the coordinates of each element. It does this iteratively until the coordinates which best fit the measurement are found. The coordinates are then approximately centered on \((0,0)\). The resulting coordinates are written out to a JSON file which can be consumed easily by the direction finding software discussed in Chapter 6. The array geometry calculator is able to calculate coordinates for arrays of any size, it only needs a map of baseline measurements.\\

Measurements were taken for all of the baselines of the constructed array and fed into the calculator. A top down view of the real array as well as a plot of the calculated geometry produced by the script is show in \Cref{fig:rf-front-end:array-coordinates}. It can be seen that the plotted array matches the geometry of the real one, indicating that the calculator is working as expected.

\begin{figure}
  \includegraphics[width=0.4\textwidth]{array-top-view}
  \includegraphics[width=0.59\textwidth]{array-coordinates}
  \caption{Top view of constructed deformed array with a plot of the calculated positions for each element next to it.}
  \label{fig:rf-front-end:array-coordinates}
\end{figure}

\subsection{Ambiguity Performance}


\begin{figure}
  \centering
  \begin{subfigure}{\textwidth}
    \centering
    \includegraphics[width=0.90\textwidth, clip=true, trim = 0 14 50 0]{4}
    % left, bottom, right, top
  \end{subfigure}
  \begin{subfigure}{\textwidth}
    \centering
    \includegraphics[width=0.90\textwidth, clip=true, trim = 0 14 50 0]{4-deformed}
  \end{subfigure}
  \caption{Ambiguity plots for 4 element square and deformed arrays with reference signal arriving at \SI{0}{\degree} showing improved ambiguity after array deformation.}
  \label{fig:antenna-array:deformed-vs-normal-over-frequency}
\end{figure}

\begin{figure}
  \centering
  \includegraphics[width=0.7\textwidth, clip=true, trim = 0 0 60 0]{4-element-circular-ambiguity-vs-phi} \em \em
  \includegraphics[width=0.7\textwidth, clip = true, trim = 0 0 0 0]{4-element-deformed-ambiguity-vs-phi}
  % left, bottom, right, top
  \caption{Ambiguity plots for 4 element square and deformed arrays at \SI{250}{\mega\hertz} with varying reference and comparison angles showing improved ambiguity after array deformation.}
  \label{fig:antenna-array:deformed-vs-normal-over-angle}
\end{figure}

The array was deformed in order to ameliorate the ambiguity issue inherent in a square array. As mentioned before, a 4 element circular array with equidistant elements is the same as a square. In order to know whether the deformation has improved ambiguity performance, the coordinates of the real constructed array are fed into the ambiguity simulator developed in Chapter 3. This allows comparing the deformed 4 element array with the standard 4 element array. The results are plotted in \Cref{fig:antenna-array:deformed-vs-normal-over-frequency} and \Cref{fig:antenna-array:deformed-vs-normal-over-angle}.

The two figures show that ambiguity has to a large extent been reduced by the deformation, especially at the centre frequency of the folded dipoles, \SI{250}{\mega\hertz}. The dark blue points have mostly been eliminated, meaning that angles that were before very difficult or impossible to tell apart are now exhibit more distinct baseline phase shifts. This is at the expense of slightly non-uniform performance and less distinct peaks in phase difference (seen by dark reds) in some directions. This is a good trade-off and the 4 element array is considered useable for the project.\\

An additional measurement of the performance of the various array configurations was done by simulating how susceptible the accuracy of the angle of arrival calculation is to corrupting noise on the input phase measurement. The simulation involves getting the ideal visibility phase shift for each visibility for a given real angle, then corrupting all of the visibilities by some RMS error and re-running the resulting corrupted phase shifts thought the DF algorithm. The output angle is then compared with the original angle. This is done for all angles over \(2\pi\) radians and the RMS error of the output angle calculated. This is done for a range of input corruption from 0 to 1 radians, for antenna configurations of 3, 4, 5, 6 and 7 elements as well as for the 4 element deformed configuration with has been constructed here. A script to perform this simulation was written\footnote{\url{https://github.com/jgowans/phase_ambiguity/blob/master/simulate_df_error_vs_visibility_error.py}} on top of the original simulator and array modeling package, and the results can be seen in \Cref{fig:antenna-array:configuration-vs-input-error-vs-output-error}. From the figure we observe that the even numbered array configurations (4 and 6 elements) have a non-zero error right from the start due to the points of total ambiguity which they exhibit. The 3 element array becomes corrupted by noise very quickly and increases fast due to not having as much information input as the larger arrays. The most interesting observation is that the 4 element deformed array performs only slightly worse than the 5 element array. This observation further indicates that the design decision of deforming a 4 element array rather than going for a 3 element array was a correct decision.

\begin{figure}
  \centering
  \includegraphics[width=0.9\textwidth]{visibility-error-vs-df-error}
  \caption{Susceptibility of different array sizes to input error. The line labeled \texttt{4'} is represents the deformed 4 element array that has been constructed.}
  \label{fig:antenna-array:configuration-vs-input-error-vs-output-error}
\end{figure}
