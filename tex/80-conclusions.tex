\chapter{Conclusions}
\label{ch:conclusions}
\graphicspath{{./img/conclusions/}}

This project has detailed the systematic design and construction of a direction finding system, able to locate both strong impulsive signals and weak narrow band signals. 

The project began with defining the problem statement and gathering user requirements in \Cref{ch:introduction}. Some notable user requirements of the DF system are that it should have a \SI{360}{\degree} field of view, it need not concentrate on a specific frequency band but rather should prototype the system and DF algorithms, and it should use hardware and software that fit into the SKA ecosystem. 

A review of DF techniques was then done in \Cref{ch:lit-review} and it emerged that the most suitable techniques would be correlative phase interferometry for the narrow band signals, and time difference of arrival (TDoA) for impulsive signals. Phase interferometry was selected as it can make use if the processing power of the SKA hardware to produce digital gain and see signals below the noise floor of the DF system. TDoA was selected due to its robustness to be able to DF any sort of strong broadband pulse.

A block level system design was done in \Cref{ch:system-design}. It showed how the system would be subdivided into distinct subsystems. First, an antenna array would be used to pick up RFI signals which would go through a RF front end for filtering an amplification. Next, digitisers in the ROACH would get the signal into the FPGA where the first stage of high-speed DSP would be done. Finally, the processed signals would be read out of the ROACH onto a computer where the angle of arrival  estimation algorithms would run, and which would act as a user interface to the system. This modular design would allow the system to be modified easily in future to, for example, upgrade the antenna array and RF front end to work at a higher frequency that would overlap with the MeerKAT telescope. \Cref{ch:system-design} also looked at result of simulations to compare the phase ambiguity of circular arrays with different numbers of elements. It was shown that in general the more element the array has the lower the ambiguity error, but also that circular arrays with odd numbers of elements are significantly better than even numbers.

Construction of the system began in \Cref{ch:rf-front-end} with the antenna array which was built from four dipoles with a center frequency of \SI{250}{\mega\hertz}. Generally even numbered array should be avoided, however it was found that by deforming the four element circular array so that it was no longer a proper circle the ambiguity issue was improved significantly. The performance approached that of a five element array. In general, for phase-based direction finding it's best to have a smaller array (elements less than a wavelength apart) to minimize ambiguity. However, for TDoA it's best to have a large array, to make the propagation delays between the elements more noticeable in the time domain. A compromise was made here to find dimensions that worked for both, but the fact that these two requirements are at odds with each other indicates that trying to design a signal system to DF both narrow band an impulsive signals will lead to worse performance than if the systems were independent. 

\Cref{ch:rf-front-end} also discussed the RF front end which was built, consisting of four individual RF chains. Each has an amplifier and low pass anti-aliasing filter. Profiling of these RF chains showed that they were relatively well phase matched and provided the expected gain of \SI{20}{\dB}.

The next phase of system implementation was the FPGA firmware design for the ROACH, detailed in \Cref{ch:firmware-design}. There are two main signal paths in the FGPA, one for the frequency domain signals and one for time domain. The frequency domain path contained a Fourier transform, spectrum cross-multiplier, accumulators and snapshot blocks. The accumulators meant that many cross-correlations to be added together, allowing weak correlated signals to stand out of uncorrelated noise. The time domain path contained a power detector and high-capacity snapshot block, able to detect and capture impulses then they occur. The lack of overlap in these DSP paths (both in how processing is done and how detections are made) further indicates that coupling both narrow band DF and impulsive DF in a single system does not have significant advantages over running independent systems. Nevertheless, lab testing which emulated environmental signals demonstrated that both DSP paths worked well. The frequency domain path successfully demonstrated using accumulation to reduce gain, and the time domain path demonstrated detection of impulses. The control and monitoring logic in the FPGA made it a general purpose runtime programmable system.

The final part of design was done in \Cref{ch:software-design} where the software interface and direction finding algorithms were implemented. The software was structured in such a way to make if easy to adapt to correlators of different configurations, be it number of elements or frequency range. The direction finding algorithms work by comparing the real observed signals with simulated signals from the antenna array model, finding the simulated angle which best matches the observed signals. This approach is general purpose as any array configuration can be simulated. Additionally, the software catered for injecting calibration factors to offset analogue mismatches.

With the full system now complete, \Cref{ch:field-trials} showed results from field trials. The DF system was made to track various sources, and its results compared with a GPS tracker. It was shown that in general the DF system tracked the sources very well, despite being in a hostile RF environment. The track of the impulsive source was initially not good due to the presence of many strong correlated signals, but after the recorded signals were cleaned up in post-processing the tracking was significantly better.

Overall, this project successfully developed and tested a full direction finding system, from user requirements to field trials. Along the way researching into antenna ambiguity, RF chains, ADCs, FPGAs and modular software design was done. Although the system managed to perform well in both narrow band and impulsive DF, these two requirements were often at odds with each other. For future work, when the systems' operating frequency range is extended into the gigahertz for MeerKAT support, the recommendation from the author is to decouple the requirements of narrow band and impulsive DF into two distinct systems.

