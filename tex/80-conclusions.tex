\chapter{Conclusions}
\label{ch:conclusions}
\graphicspath{{./img/conclusions/}}

This project has detailed the systematic design and construction of a direction finding system, able to locate both strong impulsive signals and weak narrow band signals. 

The project began with defining the problem statement and gathering user requirements in \Cref{ch:introduction}. Some notable requirements of the DF system are that it should be able to locate both narrow-band continuous signals as well as broad-band impulses, should have a \SI{360}{\degree} field of view, should use hardware and software that fit into the SKA ecosystem, and that the system be a prototype with extensible DF algorithms rather than concentrate on any specific frequency band.

A review of the relevant mathematics and DF techniques from prior literature was then done in \Cref{ch:lit-review}. It emerged that the most suitable techniques would be correlative phase interferometry for narrow-band signals, and time difference of arrival (TDoA) for impulsive signals. Phase interferometry was selected as it would be able to make use of the processing power of the SKA hardware to produce digital gain and see signals below the noise floor of the DF system. TDoA was selected due to its robustness in being able to DF any sort of strong broadband pulse.

A block level design for our system was drawn up in \Cref{ch:system-design} to show how the system would be subdivided into the following distinct subsystems. First, an antenna array would pick up RFI signals which will pass through an RF front end for filtering an amplification. Next, digitisers in the ROACH get the signals into the FPGA where the first stage of high-speed DSP is done. Finally, the processed signals are read out from the ROACH onto a computer. It is here that the Angle of Arrival (AoA) estimation algorithms are run, and where the user will interface with the system. This modular design would allow the system to be modified easily in future to, for example, upgrade the antenna array and RF front end to work at a higher frequency that would overlap with the MeerKAT telescope's band of interest. \Cref{ch:system-design} also looked at result of simulations which were done to compare the phase ambiguity of circular arrays with varying numbers of elements. The outcome was that, in general, the more element the array has the lower the ambiguity error. It also showed that circular arrays with odd numbers of elements have significantly less phase ambiguity than those with even numbers of elements.

Construction of the system began in \Cref{ch:rf-front-end} with the antenna array nd RF front end. The array was built from four dipoles with a center frequency of \SI{250}{\mega\hertz}. Generally circular arrays with even numbers of elements should be avoided, however the hardware being here was limited to only four simultaneous ADC inputs. Despite this, it was found that by deforming the four element circular array so that it was no longer a proper circle, the ambiguity issue was improved significantly. The performance approached that of a five element array. For phase-based direction finding it's best to have a compact array (elements spaced less than a wavelength apart) to minimize ambiguity. However, for TDoA it's best to have a mode widely spaced array to make the propagation delays between the elements noticeable in the time domain. A compromise was made here to find dimensions that worked for both, but the fact that these two requirements are at odds with each other indicates that trying to design a signal system to DF both narrow band an impulsive signals will lead to worse performance than if the systems were independent. \Cref{ch:rf-front-end} also discussed the RF front end which was built, consisting of four individual RF chains. Each has an amplifier and low pass anti-aliasing filter, along with cabling. Profiling of these RF chains showed that they were relatively well phase matched and provided the expected gain of \SI{20}{\dB}.

\Cref{ch:firmware-design} details the next phase of the system implementation, the FPGA firmware design for the ROACH. There are two main signal paths in the FGPA: one for the frequency domain signals and the other for time domain signals. The frequency domain path contained a Fourier transform, spectrum cross-multiplier, accumulators and snapshot blocks. The Fourier transform has the ability to receive multiple inputs simultaneously, and allows phase differences at a given frequency to be easily read. Accumulators allow for many cross-correlations to be added together, enabling weak correlated signals to stand out from uncorrelated noise. The time domain signal path contained a power detector and high-capacity snapshot block able to detect and capture impulses when they occur. Lack of overlap in these DSP paths (both in how processing is done and how detections are made) further indicates that coupling narrow band DF and impulsive DF in a single system does not have much advantage over running independent systems. Nevertheless, the lab testing which was done demonstrated that both DSP paths performed well. The lab testing emulated environmental signals by introducing configurable levels of uncorrelated noise in each channel, and generating both correlated narrow band and impulsive signals for capture and processing. The frequency domain path successfully demonstrated using accumulation to produce gain, and the time domain path demonstrated detection of impulses. The control and monitoring logic in the FPGA made it a general purpose runtime programmable system.

The final part of the design and implementation was done in \Cref{ch:software-design} where the software interface and direction finding algorithms were written. The software is a Python package which has been structured in such a way to make if easy to adapt to correlators of different configurations with regards to number of elements or frequency range. The direction finding algorithms work by first building a model of the theoretical response of the array for every angle, and then comparing the real observed signals with the model, finding the angle whose simulated response best matches the observed signals. The response is baseline phase shifts in the case of narrow band direction finding, and time delays in the case of impulse direction finding. This approach is general purpose as any array configuration can be modeled. Additionally, the software catered for injecting calibration factors to offset analogue mismatches. Measurements on the hardware were taken to find calibration factors, and tests showed that calibration successfully lowered the observed error.

The last Chapter was \Cref{ch:field-trials} which showed results from two field trials that were carried out. The first trial was early in the project with only a rudimentary two-element setup to test capture of impulses. It got useful data on the characteristics of the pulses, but concluded that more elements were needed to reliably track impulsive signals. The final fields trials on the full DF system involved having it track various transmitters and comparing its results with a GPS logger which had been running simultaneously. It was shown that in general the DF system tracked the sources very well, despite being in a hostile RF environment. The track of the impulsive source was initially not good due to the presence of many strong correlated signals, but after the recorded signals were cleaned up in post-processing the tracking was significantly better. This demonstrated that the algorithms and implementation were able to meet the requirements.

Overall, this project successfully developed a full direction finding system that met with user requirements. It was successfully tested both in the lab and in the field. Systems design was done throughout, along with research into antenna ambiguity, RF chains, ADCs, FPGAs and modular software design. The system was designed to be flexible and reconfigurable to meet the eventual frequency requirements for the SKA. Although the system managed to perform well in both narrow band and impulsive DF, these two requirements were often at odds with each other. They will conflict even more at higher frequencies when the antenna array becomes more compact. For future work, when the systems' operating frequency range is extended into the gigahertz for MeerKAT support, the recommendation from the author would be to decouple the requirements of narrow band and impulsive DF into two distinct systems.
