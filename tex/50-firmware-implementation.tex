\chapter{Firmware Design}
\label{ch:firmware-design}
\setsvg{svgpath=./img/firmware/}
\graphicspath{{./img/firmware/}}

Here are the steps which were done to produce the hardware for testing purposes.
Discussion emphasis is on what I changed or modified. Blocks which were already in existance are discussed in less detail.

\section{Software Setup}
CentOS which is the open source version of RHEL. Computer with at least \SI{8}{\giga\byte} of memory as the compile process is memory intensive. 
Number of CPU cores is not important as the Xilinx compile process only runs single threaded.
Xilinx SysGen 14.7.
MATLAB R2012B. 
Casp

\section{ADCs}
The ADCs which were available for use were the CASPER iADCs. 
These are 8-bit, dual core ADCs, where each core runs at \SI{800}{\mega\hertz}. The cores can either be interleaved to sample a single antenna at \SI{1600}{\mega\hertz} or 2 antennas at \SI{800}{\mega\hertz} each.

\section{Polyphase Filter Bank}
Consists of a polyphase FIR filter which applies a window to the input signal in order to prevent spectral leakage followed by a FFT block. FFT consumes most resources and thus some optimisations had to be done to it. 4K PFB. FFT was a real FFT block meaning it only outputs the upper half spectrum as the lower half is the same due to input signals being real. 

Shifting schedule set by software. Bit growth occurs at each stage. If the output of a stage is not shifted down by 1, it risks overflowing. However, if shifting is done unnecessarily, dynamic range is reduced as lower bits are thrown away. Algorithm coded to find optimal shifting. Discuss algorithm here.

\section{Cross Multiplier}
After the FFT, each antenna combination is multiplied together, one being the original signal and one being the complex conjugate. This is somewhat equivalent to dividing the complex numbers, where the key output is that the phase difference between the two antennas is produces. Some maths here to show that this is true. 

Optimisations done here: these are fairly large multiplier. Each pair of antennas requires an 18 bit multiplier for the real and imag components, for both simultanious channels. This means 4 18-bit multipliers for 10 combinations. 40 x 18-bit multipliers is a lot of hardware! 
To mitigate this, I made a change to the complex multiplier block to allow selecting of DSP48E for multipliers. This change was committed back into the centeral code repo for all to use.

Output of a 18\_17 x 18\_17 is a 37\_34. 

\section{Vector Accumulator}
The output vector (2K complex elements) is accumulated by summing each element. 
This is accumulated to a 48 bit number, hence allowing for substantial growth. 
This is key to getting a very good phase difference approximation as uncorrelated noise is integrated out. 
The vector accumulator is implemented by two DSP48E blocks, one for the real and one for the imag components. 
This is followed by a bram which stores and feeds back the vector to the DSP48E adder. 

The design is such that 48 bits are continuously accumulated. After the accumulation has run for a configurable number of iterations, the most significant 32 bits are sliced off and snapped. By accumulating 48 bits, no data is thrown away until the snap. Commit XXXXX makes this change to the dsp48\_bram\_vacc block in the casper library.

\section{Compiling the design}
This design was iterated on and compiled more than 100 times as functionality was added and experimentation was done.
The design is very large, using over 95\% of the available logic on the Virtex 5 FPGA. This has meant that it is a difficult design to compile, often failing to meet timing requirements.
Running at \SI{200}{\mega\hertz} means that a signal needs to get from the data source element to the data end element through any intermediate logic within \SI{5}{\nano\second}. 
There are two components to the propogartion time of a signal inside the FPGA. First is propogation delay through logic and the second is propogation delay through routing. When the design is very large, routing delay goes up as logic gets spread further apart on the chip. When there are lots of gates which a signal has to go through, logic delay goes up. 

The data signal needs to arrive at the data end element a short time before the clock signal, and it needs to remain valid for some short time after the clock edge. This first amount of time is known as setup time and the time after the edge is known as hold time. If the data propogration delay is too large, setup will not be met. If clock skew is too large, hold time will not be met.
When timing fails, Xilinx Sysgen reports which net has failed, between which source and end elements timing has failed, which logic blocks the data line passed through, where they are physically located on the FPGA, how much of the delay was caused by logic propagation and how much routing propagation. 

If the source and destination elements are far apart, timing will be dominated by routing. The designed may attempt to use tools like FloorPlanner to move the elements around on the chip to get them close together. For a design like this using about 90\% of the logic, there is almost no room to adjust and this becomes very challenging. 

Solution: inserting a dummy latch in the data path so that the data will be latched in somewhere between the original source and destination elements. This is essentially increasing the latency of the pipeline. This is not a easy solution though: often for logic constrained designed, adding a latch to a wide data pipeline will make even more data paths fail timing as there is not less logic available for optimizing logic placement. It's necessary to carefully consider exactly where and why timing is failing before attempting to fix. Trying to fix timing errors is painful and slow, considering that a compile can take over 2 hours before throwing a timing error. 

\input{./tex/52-firmware-frequency-correlations}
\input{./tex/55-impulse-capture}
\input{./tex/57-simulations}
\input{./tex/58-firmware-performance-measurements}
