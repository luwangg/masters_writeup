\chapter{RF Front End}
\label{ch:rf-front-end}
\setsvg{svgpath=./img/rf-front-end/}
\graphicspath{{./img/rf-front-end/}}

This chapter details the design and characterisation of the chain of antenna, low noise amplifier (LNA), lower pass filters and cabling. The cabling is important as each RF chain needs to have exactly\footnote{Some phase difference between the RF chains can and will be calibrated out, but the calibration routine is most robust when the RF chain phases are already closely matches} the same phase delay.

The LNAs which are used are the ZLF-500HLN from MiniCircuits. This part operates from \SI{10}{\mega\hertz} to \SI{500}{\mega\hertz} which is ideal for the application. The gain is approximately 21 dB across this band acording to the datasheet. The input current is a maximum of \SI{110}{\milli\ampere}.

\section{Portable Configuration}
The RF front end needs to be able to be taken out into the field and used while running on batteries.
Two ZIPPY Compact \SI{1000}{\milli\ampere\hour} 3S 25C battery packs and 4 LM7815 regulators were acquired. 
Each amplifier has its own regulator to try to reduce any electrical coupling between the regulators through their power rails.
The battery packs output \SI{12.5}{\volt} when fully charged meaning a combined input to the \SI{15}{\volt} regulators of \SI{25}{\volt}. Hence the regulators are dropping \SI{10}{\volt} at \SI{100}{\milli\ampere} or \SI{1}{\watt}.
The LM7815s can handle this dissipation provided a heat sink is connected, which was done.
This power distribution circuitry was put onto veroboard.

All of the amplifiers and filters and the power distrubution board were mounted to a wooden board.
Care was taken to make sure that the low pass filters were securely and firmly attached to the board as there is a risk that the SMA connector would snap off if they they were bumped. 
The circuit diagram for this board and the resulting hardware implementation for this RF front end are are shown in figure ??.

\section{System Temperature Calculations}
Tsys.
Channel bandwidth = 390 kHz. 
Noise figure: 3.9 dB. 
LNA gain: 20 dB
LPF insertion loss: 1 dB
connector and cable losses: 2 dB

\begin{figure}
  \centering
  \begin{subfigure}{\textwidth}
    \centering
    \includesvg[width=\textwidth]{fr-front-end-circuit-sch}
    \caption{Schematic}
  \end{subfigure}\\[1em]
  \begin{subfigure}{\textwidth}
    \centering
    \includegraphics[width=0.6\textwidth]{rf-front-end-charging}
    \caption{The thing}
  \end{subfigure}
  \begin{subfigure}{\textwidth}
    \centering
    \includegraphics[width=0.6\textwidth]{lpf-in-holder}
    \caption{LPF in holder}
  \end{subfigure}
  \caption{RF Front end design and construction}
\end{figure}


\section{Antenna Array}

Some antennas.

\begin{figure}
  \begin{subfigure}{\textwidth}
    \centering
    \includegraphics[width=0.85\textwidth]{antenna-array-on-roof}
  \end{subfigure}\\[1em]
  \begin{subfigure}{\textwidth}
    \centering
    \includegraphics[width=0.85\textwidth]{antenna-s11}
  \end{subfigure}
  \caption{Array of four FD-250 folded dipoles being measured and the result of the S11 measurements showing acceptable performance between \SI{200}{\mega\hertz} and \SI{300}{\mega\hertz}.}
\end{figure}

\section{LNA and LPF Measurements}
As discussed in lit review, Schleher, it is necessary for an interferometry system to have its RF front end components matched. The front end of this system comprises an antenna, a cable from the antenna to a anti aliasing filter, the filter is plugged into an LNA and a cable runs from the LNA to the digitiser. 
It's fairly strightforward to ensure phase matching between cables. Provided the cables are of the same type, their velocity factors will be equal and it is hence only necessary to ensure that they are of the same length.
The LPFs and LNAs however may have phase mismatches due to manufacturing tolerances. 
There are two areas of interest:
\begin{enumerate}
  \item Does the LPF and LNA provide sufficient attenuation above the Nyquist frequency to ensure that no aliasing occurs?
  \item Hope closely are the phases of the sets of LPFs and LNAs matched to each other?
\end{enumerate}

\section{Anti Aliasing}
As per (cite ADC datasheet and show graph here) the iADC is not receptive to signals above 3 GHz. Hence, it is only necessary to ensure that the LPF and LNA have sufficient attenuation from Nyquist frequency to 3 GHz. 

\begin{figure}
  \centering
  \begin{subfigure}{\textwidth}
    \centering
    \includegraphics[width=0.95\textwidth]{zfl500-10-3000}
    \caption{Gain from 0 to \SI{3}{\giga\hertz} in dB. Antialisaing is sufficient}
  \end{subfigure}\\[1em]
  \begin{subfigure}{\textwidth}
    \centering
    \includegraphics[width=0.95\textwidth]{zfl500-10_500-mag-db}
    \caption{Gain from 0 to \SI{500}{\mega\hertz} in dB. Useable seems to be up to \SI{350}{\mega\hertz}}
  \end{subfigure}
  \caption{VNA measurements of S21 of RF chains showing antialising properties}
\end{figure}
\begin{figure}
  \centering
  \includegraphics[width=\textwidth]{zfl500-10-500-phase}
  \caption{Phase from 0 to \SI{500}{\mega\hertz} showing very good phase matching over operating frequency range (below \SI{350}{\mega\hertz}}
\end{figure}

Phase RMS errors (degrees)
0x1: 0.58
0x2 0.64
0x3: 2.3

