\section{Report Outline}
Chapter 1 has so far contained the problem statement and user requirements of the system which needs to be designed and implemented. 

\Cref{ch:lit-review} explores the existing literature around DF systems. This serves as the foundation for the system being designed here. The exploration focuses on direction-finding fundamentals and on a comparison of algorithms. It examines DF for both continuous narrowband signals and impulsive signals. It explores detection algorithms for impulsive sources, as well as calibration techniques and error calculations. The outcome of this chapter should be the gathering of sufficient information to make informed system design decisions.

\Cref{ch:system-design} contains the high level system design, showing what blocks need to be designed and how they will be linked together in order to build a complete system. Here, the system is broken into three main components: the RF front end, the ROACH digitiser and DSP engine, and the computer software for final AoA computation and logging. Additionally, the exact DF algorithm being used by the system will be defined and simulations of its performance run.

\Cref{ch:rf-front-end} discusses the antenna array and RF front end. It details designing and building of a 4-element antenna array, as well as the assembling and measuring of a board of low noise amplifiers and filters.

\Cref{ch:firmware-design} discusses the implementation of the FPGA firmware to do both data acquisition and the first stage of DSP. The focus is on designing the FPGA sub-system in simulink and then demonstrating it being functional both in simulink simulations and, most importantly, running on hardware in the lab to show successful processing of real signals.

\Cref{ch:software-design} details Python code which implements the direction-finding algorithms. It requires interfacing the computer with the ROACH to read out data, creating a model of the antenna array, and estimating the AoA based on the received data and the array model. It also has to handle calibration, logging, and showing signal plots to give the operator insight into the signals being received.

\Cref{ch:field-trials} reports on the field trials of the DF system. There are two field trial sessions: one early in the project with a basic system to get a feel for the types of RFI data. Another with the final system doing real DF measurements and checking the accuracy of the system.

\Cref{ch:conclusions} contains conclusions indicating what has been achieved in the project and comments on the system which has been implemented. Also, scope for future work is looked at.
