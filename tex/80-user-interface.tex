\chapter{User Interface}
\label{ch:user-interface}

The UI is modeled on that presented in \cite{guerin2012passive}.

\section{Software Architecture}
The user interface has been implemented in the form of a web interface, as opposed to an separate application. Although a separate application would have been able to provide more functionality and better graphics, a web-based app was defined in the user requirements due to it being much easier for a user to connect to the instrument if it has a web interface rather than the user having to install various software packages and attempt to run programs.

As the web app needs to interface heavily with the Python code developed earlier to implement the direction finding, it makes sense to utilise a framework which makes this interaction easy. There are many popular Python web frameworks, including Django, Flask and Pyramid, Tornado, Falcon and others. The one which has been selected for this work Python's Pyramid web framework. It's a light-weight, fast and flexible framework which leaves most of the implementation details up to the developer. Also, Pyramid has been used in SKA before for other project so there is internal expertise around it. 



