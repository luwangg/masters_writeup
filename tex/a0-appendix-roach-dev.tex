\chapter{ROACH Development}
\label{appendix:roach-development}
\graphicspath{{./img/roach-dev/}}

\section{Compiling}
Compiling is rough, bro.
PlanAhead: \url{https://github.com/casper-astro/mlib_devel/commit/a949c9d}
Optimizing with planahead: \url{https://casper.berkeley.edu/wiki/Speed_Optimization_with_PlanAhead}

\section{Vector Accumulators}
I did make this.

Convert: attempts to maintain the value but changes the binary representation. Signed vs unsigend or number of fractional bits. 

Reinterpret: forces the output to a specific data type with no regard for value. Leaves the binary value untouched but changes how that binary value is interpreted. Number of fractional bits or signed vs unsigned. 

DSP48E vacc deals in 48-bit signed values. Up to user to reinterpret the output to match the input. 

Feeding in a tone right in the middle of a frequency bin (specifically 123.047 MHz) with a 1 second accumulation time, the VACC overflows with a tone 20 bits peak to peak.
The vacc grows linearly and linearly with the \emph{power} of the input signal because it's accumulating the cross correlation which is a complex multiplication of amplitudes. 
Hence at ADC full scale, the vacc will have increased by \(\frac{256}{20}^2 = 163\) times its value.
Hence the accumulation time should be no longer than \SI{6}{\milli\second} to ensure that at maximum ADC input no overflow will occur. This corresponds to a length of 2400 accumulations.
Because: 2K point FFT. 1K real points. 2 output points per FPGA clock cycle. Hence 512 cycles per vacc.
\(0.006 = \frac{x}{512}\) 

Each vacc takes \(1 / (200e6 / 512)\) seconds. 400 000 vaccs will hence take \(400e3 / (200e6 / 512) = 1 \)

\section{Using the DRAM}
DRAM needs to run faster than fabric. Runs at \SI{266}{\mega\hertz}. Difficult routing. DRAM design significantly harder to compile. 
Dram stores samples: a a a a b b b b c c c c d d d d a a a a b b b b etc. 
How to access: naive is to iterate through each element and assign it based on the index.
Much faster to use fancy numpy. The key is the flexability of the numpy array which has a reshape method which can change the dimensions of an n-dimensional aray to any other dimensions so long as the number of elements is preserved. 
Here: compare run time of original method vs numpy. Show numpy code.

\begin{landscape}
  \thispagestyle{empty}
  \begin{figure}
  \centering
  \makebox[\textwidth][c]{
    \includegraphics[width=1.3\paperwidth]{frequency-domain-path}
  }
  \caption{Signal flow through frequency domain processing chain. Two dual ADCs into 2K point FFT into cross correlator into vector accumulators into snapshots. Note how the bus widths increase from 8 to 36 to 148 to 256 bits.}
  \label{fig:roach-dev-frequency-domain-chain}
  \end{figure}
\end{landscape}

\section{ADCs}
The ADCs which were available for use were the CASPER iADCs. 
These are 8-bit, dual core ADCs, where each core runs at \SI{800}{\mega\hertz}. The cores can either be interleaved to sample a single antenna at \SI{1600}{\mega\hertz} or 2 antennas at \SI{800}{\mega\hertz} each.

\section{Polyphase Filter Bank}
Consists of a polyphase FIR filter which applies a window to the input signal in order to prevent spectral leakage followed by a FFT block. FFT consumes most resources and thus some optimisations had to be done to it. 4K PFB. FFT was a real FFT block meaning it only outputs the upper half spectrum as the lower half is the same due to input signals being real. 

Shifting schedule set by software. Bit growth occurs at each stage. If the output of a stage is not shifted down by 1, it risks overflowing. However, if shifting is done unnecessarily, dynamic range is reduced as lower bits are thrown away. Algorithm coded to find optimal shifting. Discuss algorithm here.

\section{Cross Multiplier}
After the FFT, each antenna combination is multiplied together, one being the original signal and one being the complex conjugate. This is somewhat equivalent to dividing the complex numbers, where the key output is that the phase difference between the two antennas is produces. Some maths here to show that this is true. 

Optimisations done here: these are fairly large multiplier. Each pair of antennas requires an 18 bit multiplier for the real and imag components, for both simultanious channels. This means 4 18-bit multipliers for 10 combinations. 40 x 18-bit multipliers is a lot of hardware! 
To mitigate this, I made a change to the complex multiplier block to allow selecting of DSP48E for multipliers. This change was committed back into the centeral code repo for all to use.

Output of a 18\_17 x 18\_17 is a 37\_34. 

\section{Vector Accumulator}
The output vector (2K complex elements) is accumulated by summing each element. 
This is accumulated to a 48 bit number, hence allowing for substantial growth. 
This is key to getting a very good phase difference approximation as uncorrelated noise is integrated out. 
The vector accumulator is implemented by two DSP48E blocks, one for the real and one for the imag components. 
This is followed by a bram which stores and feeds back the vector to the DSP48E adder. 

The design is such that 48 bits are continuously accumulated. After the accumulation has run for a configurable number of iterations, the most significant 32 bits are sliced off and snapped. By accumulating 48 bits, no data is thrown away until the snap. Commit XXXXX makes this change to the dsp48\_bram\_vacc block in the casper library.
