\chapter{Field Trials dimensions}
\label{appendix:field-trials-dimensions}
\setsvg{svgpath=./img/field-trials/}
\graphicspath{{./img/field-trials/}}

Two receives: RxA and RxB
RxB \SI{1}{\meter} futher away. 
Transmitter: \(s(t)\)
Direct:
\begin{equation}
  r(t) = f_a(\theta_a)f_b(\theta_b)(s(t)\frac{\lambda}{4\pi R} \exp\left\{ -j 2 \pi \frac{R}{\lambda}\right\}
\end{equation}
Where \(f_a(\theta_a)\) and \(f_b(\theta_b)\) are the beam patterns of antenna \(a\) and \(b\) respectivly. For folded dipoles, we will assume them to be cosine beams. For this reason it makes sense to keep t Tx and Rx antennas at the same elevation as it will allow the direct beam to be in the peak of the mai lobe

The following simulation was done: one Tx antenna emitting some signal \(s(t)\), and two receiving antennas, RxA and RxB with RxA being some radius \(R\) from Tx and RxB being \SI{1}{\meter} further from the source than RxA. There are two signals paths, one from the direct beam, \(R\) and one from the reflected beam \(R^\prime\). A reflected beam has a ground reflection coefficient \(\rho e^{j \phi}\)
As per Collin, Antenna and Radio Wave Propagation.
\begin{equation}
  \rho e^{j\phi} = \frac
    {(\kappa - j\chi)\sin\theta - \sqrt{(\kappa -j\chi) - \cos^2\theta}}
    {(\kappa - j\chi)\sin\theta + \sqrt{(\kappa -j\chi) - \cos^2\theta}}
\end{equation}
where\(\chi = \alpha/\omega\epsilon_0\) with \(\alpha\) as the soil conductivity and \(\omega\) being angular frequency, \(\kappa\) the soil dielectric constant and \(\theta\) grazing angle

Now we have \(r^\prime(t)\) which is the received signal as a result of the longer path length: \(R^\prime\) as well as the ground reflection coefficient: \(\rho e^{j\phi}\)
\begin{equation}
  r^\prime(t) = \rho e^{j\phi} s(t)\frac{\lambda}{4\pi R^\prime} \exp\left\{ -j 2 \pi \frac{R^\prime}{\lambda}\right\}
\end{equation}

The final signal arriving is the combination: \(r(t) + r^\prime(t)\).
Note in the worst case, where \(\theta\) is very small (large distance, small elevation), \(R \approx R^\prime\) and \(\rho e^{j\phi} \approx -1\). This equates to complete destructive interference and should be avoided.

We want to know what range and elevation would produce results which suffer least from ground reflection interference. Most critically, we want the phase of combined signal to be as close as possible (within a few percent error) of the ideal case. 
Assumptions: Tx and Rx heights are the same. The distance between RxA and RxB is \SI{1}{\meter}. This is representative of the built array. Simulate for \(h \in \left[ 0.5 , 5 \right]\) and \(R \in \left[5, 50\right]\)
Algorithm:
\begin{enumerate}
  \item For a given R
    \begin{enumerate}
      \item compute the ideal received signal for RxA and RxB with no ground reflection. Friis only.
      \item Store this ideal amplitude and phase difference.
      \item for a given h:
        \begin{enumerate}
          \item compute \(\theta\), the grazing angle and \(R^\prime\) the longer path length
          \item compute the resulting \(\rho e^{j\phi}\) from \(\theta\)
          \item compute the received signal from the reflection which is a function of the path length multiplied by the ground reflection coefficient. 
          \item add the direct and reflected.
          \item add a data point to the plot: difference between ideal and actual amplitude. difference between ideal and actual phase.
        \end{enumerate}
    \end{enumerate}
\end{enumerate}

For folded dipoles, we will assume them to be cosine beams. For this reason it makes sense to keep the Tx and Rx antennas at the same elevation as it will allow the direct beam to be in the peak of the main lobe



\begin{figure}
  \centering
  \begin{subfigure}{\textwidth}
    \centering
    \includesvg[width=\textwidth, svgpath=./img/field-trials/]{ground-reflection-model}
    \caption{Model being simulated. Ideal case is the phase difference between receiving elements from \(ra\) and \(rb\) paths only. Real case is phase difference due to all four paths with ground reflection coefficient \(\rho e^{j\phi}\)}
  \end{subfigure}\\[2em]
  \begin{subfigure}{\textwidth}
    \centering
    \includegraphics[width=\textwidth, clip=true, trim=0 0 140 0]{multipath-phase-shift}
  \end{subfigure}\\[2em]
  \begin{subfigure}{\textwidth}
    \centering
    \includegraphics[width=\textwidth, clip=true, trim=0 0 140 0]{multipath-amplitude-shift}
     % left, bottom, right, top
  \end{subfigure}
  \caption{Simulation of change in antenna array response due to ground reflection multipath}
\end{figure}

Below \SI{3}{\meter} elevation, it looks like the phase becomes fairly uniform with distance after about \SI{25}{\meter}. Above \SI{1.5}{\meter} elevation the amplitude degradation is below \SI{5}{\decibel} so no need to be concerned.

We have assumed flat wave fronts. This may not be true. Again, small sumulation to help decide on a suitable distance. Need to be far enough that the non-flatness of the wavefronts becomes not a problem.

Model: Two antennas: one directly R away from source, one d away and l to the side. l being a baseline length. Also assumed to be 1 m. 
Flat: phase diff should be 0.
Non-flat: will be some shift.
\begin{figure}
  \centering
  \begin{subfigure}{\textwidth}
    \centering
    \includesvg[width=0.7\textwidth]{curved-wavefront-model}
  \end{subfigure}\\[2em]
  \begin{subfigure}{\textwidth}
    \centering
    \includegraphics[width=0.8\textwidth]{curved-wavefront-error}
  \end{subfigure}
\end{figure}

