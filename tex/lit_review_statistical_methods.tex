\section{Statistical Methods}

As per discussion in Poisel \cite{poisel2012electronic}. 

Statistical methods generally involve taking snapshots of the signals received at each sensor, \(\vec{r}(t)\) and then calculating the covariance matrix of the received data.
It can be shown that the covariance matrix is diagonal for incoherent sources, non-diagonal and non-singular\footnote{A singular matrix is one which is not invertible, meaning the determinant is 0} for partially coherent sources and non-diagonal and singular if at least one of the sources are coherent\cite{poisel2012electronic}. 

The direction finding process involves doing eigen analysis (finding eigenvalues and eigenvectors) for the covariance matricies. 
Two sorts of eigenvectors will be produced:
\begin{description}
  \item[Noise Eigenvectors] have an eigenvalue equal to the noise power.
  \item[Signal Eigenvectors] have some property that I don't understand
\end{description}

\subsection{Subspace Methods}
This class of methods attempts to decompose the covariance matrix into eigen subspaces for noise and subspaces for signal. In some sense this equates to a beamforming algorithm which attempts to find an angle resulting in the strongest signal. 

\subsection{Maximum Likelihood}
Foobar
