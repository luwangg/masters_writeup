\section{Direction Finding Techniques}
A radio direction finder is a passive device able to ascertain the direction of the source of electromagnetic radiation. 
The general structure of a system is an antenna array feeding into a receiver feeding into some sort of processing feeding into a readout to operators.

Radio DF has been around since the start of the 1900's. Some of the first people to work on DF  systems were Marconi and Brown, you manually rotated a two element array with half-wavelength spacing to find the bearing to a ship which was broadcasting signals. 

As discussed in \cite{farrier1990direction}, many high-performance DF algorithms which have been proposed over the years have had stringent requirements for array geometry or signal environment. The requirement for this project is for the system to be versatile and reconfigurable, so no such restrictions on array geometry or signal environment or modeling accuracy must exist. 

\subsection{Classes of Direction Finder Systems}
The following overview of types of direction finding systems is based on discussions in the Electronic Warfare and Radar Systems Engineering Handbook \cite{desk1997electronic}.
In general, classes of system include:

\subsection{Scanned beam}
This is a direct amplitude technique. An antenna with gain is rotated and this causes the power output of the antenna to vary with rotation angle. This is the earliest form of direction radio direction finding (RDF) \nomenclature{RDF}{Radio Direction Finding} and was implemented in the early 1900's. One of the first uses of this RDF device was to locate the bearing to a ship which was transmitting a signal for navigation purposes. A loop antenna was frequently used in the early 1900's as it was a simple antenna to construct and the beam pattern has a sharp null. An example antenna and beam pattern are shown in \autoref{fig:lit_loop_antenna}. As the change in antenna output power per degree rotated is highest around the null, the antenna was often aligned such that the signal of interest was in the null. If multiple signals originating from different bearings are present, the system cannot operate. While today it may be possible to put highly selective receivers on the output of the antenna, this technology did not exist when scanning beam DF systems were originally used.

Note that as the beam pattern is symmetric about both the x and y axis, there are in general 4 ambiguous points. However, if the antenna can be rotated such that the signal is in the null, there is then only 2 ambiguous points, or \SI{180}{\degree} ambiguity. This can be resolved with an  additional antenna element to resolve. Note also that this is not an instantaneous DF technique as it requires time for the antenna to be rotated. Hence, this technique is not suitable for finding transients. 

Loop antennas DF systems are very sensitive to multipath errors, especially ionospheric reflection \cite{jenkins1991smallaperture}. 
\begin{figure}
  \centering
  \begin{subfigure}[b]{0.4\textwidth}
    \includegraphics[width=\textwidth]{lit_review/loop_antenna}
    \caption{Loop antenna used for direction finding in 1918. Src: \cite{grabau1989funkpeiltechnik}}
  \end{subfigure}
  ~
  \begin{subfigure}[b]{0.35\textwidth}
   \includegraphics[width=\textwidth]{./img/lit_review/loop_antenna_beam}
   \caption{Beam pattern of loop antenna or dipole. Note the sharp null. Src: \cite{jenkins1991smallaperture}}
  \end{subfigure}
  \caption{Loop antenna and beam pattern}
  \label{fig:lit_loop_antenna}
\end{figure}


\subsection{Crossed Loop}
The crossed loop technique is an amplitude comparison technique. As is shown in \autoref{fig:lit_crossed_loop_antenna}, using two loop antennas perpendicular to one another produces two beam patterns: one being proportional to the sine of the angle and the other to the cos of the angle. By comparing the signal power from each antenna, it is possible to ascertain the angle of arrival (AoA) \nomenclature{AoA}{Angle of Arrival}. This system is instantanious as only a single pulse is necessary to ascertain the ratio of antenna output power. However it has many ambiguities. Some ambiguities can be resolved with a sense antenna. For optimal ambiguity resolution, the crossed loop should be rotated such that signal of interest is located in the null of one of the loops and the peak of the other.
As with the scanned beam, the crossed loop suffers from significant performance degredation arrising from multipath, specifically ionospheric reflection. 
\begin{figure}
  \centering
  \begin{subfigure}[b]{0.3\textwidth}
    \includegraphics[width=\textwidth]{./img/lit_review/loop_antenna_crossed}
    \caption{Example of crossed loop antenna.}
  \end{subfigure}
  ~
  \begin{subfigure}[b]{0.4\textwidth}
    \includegraphics[width=\textwidth]{./img/lit_review/loop_antenna_crossed_beam}
    \caption{Crossed loop antenna beam pattern showing difference in magnitude seen by each loop. Src: \cite{jenkins1991smallaperture}}
  \end{subfigure}
  \caption{Crossed loop antenna and beam pattern}
  \label{fig:lit_crossed_loop_antenna}
\end{figure}

\subsection{Adcock Array}
The Adcock array was developed and patented in 1919 by British army engineer Frank Adcock \cite{gething1991radio}.
Instead of using loop antennas, the Adcock array makes use of two orthogonally orientated (crossed) pairs of monopole or dipole antennas. Typically a North-South pair and an East-West pair are used. These antenna pairs produce the same beam pattern as the loop antennas. 
As a loop antenna can be modeled by two vertical antennas on the sides and two horizontal antennas on the top and bottom of the loop, it should be clear that the loop is not polarisation selective. While the direct beam from the signal source may be vertically polarised, the reflected beam from atmospheric reflection was also being received and corrupting the signal output of the array.
The Adcock array which uses only vertical elements maintains the same beam pattern as the loop but is not sensitive to horizontally polarised radiation hence offers better performance for a DF system.
The antenna configuration of Adcock and Watson-Watt is shown in \autoref{fig:lit_adcock_array}.
This array configuration was studied extensively in the 1930's and played a significant role in the electronic warefare of World War II \cite{gething1991radio}.

\begin{figure}
  \centering
  \begin{subfigure}[b]{0.25\textwidth}
    \includegraphics[width=\textwidth]{./img/lit_review/adcock_model}
  \end{subfigure}
  ~
  \begin{subfigure}[b]{0.6\textwidth}
    \includegraphics[width=\textwidth]{./img/lit_review/adcock_implementation}
  \end{subfigure}
  \caption{Left: model for adcock antenna showing N-S and E-W pairs. Right: Japanese implementation of array for 2 MHz direction finding.}
  \label{fig:lit_adcock_array}
\end{figure}


\subsection{Watson-Watt Evaluation}
Ambiguity is one of the major difficulties which need to be overcome in direction finding systems. In the mid 1920's, Sir Robert Watson-Wat developed an improved DF system based on the Adcock array configuration. As discussed above, the beam pattern of the output of an Adcock array is one cosine shaped beam and one sine shaped beam. 
By exploiting the trigonometric properties of these functions and adding an additional sense element for to easily compute the angle of arrival from an array with this cos/sin property. Graphically: \autoref{fig:watson-watt}.

For a more detailed discussion of the mathematics behind the Watson-Watt DF algorithm, see Poisel \cite{poisel2008introduction}. For simulation of Watson-Watt: \cite{adcockwatsonwattrdf}.

\begin{figure}
  \centering
  \begin{subfigure}[b]{0.48\textwidth}
    \includegraphics[width=\textwidth]{./img/lit_review/watson-watt-processing-analogue}
    \caption{View showing transformer configuration to produced required difference signals. Src: \cite{poisel2008introduction}}
  \end{subfigure}
  ~
  \begin{subfigure}[b]{0.48\textwidth}
    \includegraphics[width=\textwidth]{./img/lit_review/watson-watt-processing-digital}
  \caption{Abstraced view of signal chain for digital processing. Src: \cite{rhode2000introtodf}}
  \end{subfigure}
  \caption{Watson-Watt array processing technique using the Adcock array. Note that difference channels from crossed beams are formed}
  \label{fig:watson-watt}
\end{figure}

\subsection{Amplitude Comparison}
This uses an antenna array. The antennas either have gain themselves or beam forming techniques are used to create beams with gains. Amplitude comparison techniques are in general simple to implement, but provide low resolution and low accuracy compared to phase methods due to their dependence on antenna beam pattern and their sensitivity to multipath. In practice, typical amplitude comparison systems have a DF RMS accuracy between \SI{3}{\degree} and \SI{10}{\degree}.
\subsection{Phase Interferometry}
This technique is done by comparing the phase arriving at each element of an array. In general it has high complexity, may suffer from ambiguity, but can achieve comparatively high accuracy direction measurements, often between \SI{0.1}{\degree} and \SI{3}{\degree} for real systems. In comparison to amplitude systems, phase systems are more tolerant of multipath signals. The high complexity is introduced from having to carefully match the phases of the RF chains from each array element, and having to incorporate ambiguity resolution algorithms. 

\section{Phase Interferometry}
The simples phase interferometry system is a two element array. This is shown graphically in figure X. The phase difference at the output of the system is:
\begin{equation}
\phi = \frac{2 \pi d \sin \theta}{\lambda}
\end{equation}
Two notes about this simple two element array:
\begin{enumerate}
  \item The values of the sine function of only unique between \SI{-90}{\degree} and \SI{90}{\degree}. Hence, the output of the array is only unambiguous over a \SI{180}{\degree} field of view. A two element phased interferometry array cannot resolve this ambiguity. The best it can do is use antennas with gain to reject signals from one of its sides. 
 \item If the element spacing is larger than \(\frac{\lambda}{2}\) the ambiguity gets worse. The unambiguous field of view is \(\theta = 2 \sin^{-1}(\frac{\pi}{2d})\).
 \end{enumerate}
 The solution to this ambiguity problem is to use more than two elements, thereby constructing a multiple baseline interferometer. 
